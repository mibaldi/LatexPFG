\chapter{Usuarios}
\label{ch:Usuarios}


% El siguiente es el texto que aparece en el encabezamiento del proyecto fin de grado (si se quiere que sea distinto al nombre del capítulo))
\chaptermark{PFG}

El método de trabajo usado en este proyecto se basó totalmente en los usuarios como método de creación de una aplicación. Las aplicaciones tanto para móviles como para ordenador tienen un público definido, y por ello se les tiene que dar importancia a los usuarios.
Una vez creados los casos de uso hay desarrolladores que realizan el proceso de pruebas con usuarios, y utilizan a estos para buscar fallos en su producto. El método que se quiso seguir con este proyecto es un tanto distinto, utilizar las ideas propuestas de los usuarios principalmente para facilitar al desarrollador a realizar todas las pruebas sobre la aplicación, pero no es la finalidad principal de estos. El enfoque que se les dio a los usuarios aquella vez, fue un trabajo más en equipo con el desarrollador.

Los usuarios son los que van a usar la aplicación, y los que van a tener que querer descargársela. Si los casos de uso que se implementen no están dirigidos a todo tipo de usuarios, el número de personas que se la descarguen será pequeño.
Por este motivo, los usuarios irán diciendo qué tipo de casos de uso se incluirán en las siguientes versiones de la aplicación, y cómo la construirían ellos si la fuesen a usar día a día.
El método consiste en la implementación de la aplicación en ciclos pequeños que no superen el mes de desarrollo.
Cada ciclo tendrá unos usuarios específicos, y la aplicación irá creciendo y modificándose al cabo de los ciclos, para que al finalizar, en cada uno se tenga una aplicación funcional con más casos de uso que la anterior.

Para empezar el primer ciclo, los Baldusers que tomaron parte fueron el tutor y el desarrollador.
En este ciclo se desarrolló la idea y se implementaron casos de uso simples pero funcionales.
Cuando se consiguió un producto enfocado a lo que sería la idea final, se incluyeron 4 personas nuevas. Estas personas debían ser usuarios que el desarrollador supiera qué iban a probar la aplicación y se molestarían en responderle a las preguntas. 

En este ciclo, el desarrollador se puso en contacto con esos usuarios y les hizo llegar la aplicación. Ellos la probarían y le dirían al desarrollador lo que tendría que añadir y modificar de la versión inicial. El desarrollador, aparte de escuchar las propuestas, les realizó una serie de encuestas para comprobar qué tipo de usuarios eran.
Al finalizar el ciclo con los usuarios anteriores, se escogieron los casos de uso que se iban a realizar de forma inmediata, los que se realizarían en versiones siguientes o los que se quedarían fuera del proyecto.
También se implementaron los casos elegidos que darían paso a un nuevo ciclo donde se aumentaron el número de usuarios.

En este ciclo, el tipo de usuario era muy importante, no importaba tanto la cantidad de personas sino el tipo de persona a las que se añadía para hacer las pruebas.
Se duplicaron el número de usuarios que se había escogido para la anterior, y se escogieron de 8 a 10 usuarios nuevos con distintos tipos de perfil, y se subiría una versión nueva de la aplicación con los cambios realizados.
Hubo que avisar a los usuarios antiguos que se había subido una versión nueva, y a los nuevos, se les dio acceso y se les dijo que se la descargaran.
Al finalizar el ciclo, se tuvieron nuevos casos de uso de 15 personas distintas, lo cual propicio el tener que escoger cuál se implementaría y qué modificaciones se harían.
Después se realizó una versión nueva y se subió a Google Play, y los usuarios invitados la evaluaron para encontrar fallos.



\section{Tipos de usuarios}
\label{secc:tipo de usuarios}

En Baldugenda se han estado colaborando con diferente tipos de usuarios, y se ha podido comprobar la diferencia de usar distintos perfiles de personas para desarrollar una aplicación móvil.
Hay distintos ámbitos que separan los perfiles de los usuarios, se pueden dividir por edades, también por tipos de estudios, por conocimientos sobre el tema, o por sexo.
Aparte de esas divisiones hay muchas otras, pero en el caso de Baldugenda no se precisó de más para sacar conclusiones.
La mayoría de los usuarios que probaron la aplicación fueron hombres, hubo dos mujeres en la fase de usuarios, y las diferencias de la división por sexo no se encontraron, así que no se le dieron importancia en las fases siguientes.
Hay colectivos como los discapacitados que son usuarios especiales, a los que hay que desarrollar cosas muy específicas para ellos, por este motivo y porque si se realizaba un trabajo de este estilo, hubiera aumentado muchísimo el coste. Se decidió no incluirlos en las divisiones y añadirlos dentro de las exclusiones del proyecto.
Dos divisiones que sí que se vieron muy afectadas a la hora de hacer las pruebas fueron la de la edad y los conocimientos sobre el tema, en esta situación se tiene en cuenta los conocimientos informático y manejo de dispositivos móviles.

Se tuvo que escoger el tipo de usuario que se buscaba que fuera significativo dentro de la aplicación, por ello y viendo el tipo de aplicación que se quería realizar, se decidió que el usuario final debía ser usuarios universitarios, ya que la aplicación tenía como puntos muy específicos las asignaturas y los exámenes, asuntos que van muy ligados al ámbito de los estudiantes.
Esa decisión marcó el tipo de usuario que se escogería en las primeras pruebas, las cuatro primeras personas  eran universitarios, y no se le dio importancia a sus conocimientos sobre informática ni sobre móviles, lo único que importó fue que tuvieran dispositivos compatibles, y que yo tuviera relación cercana con ellos para poder tener los resultados lo más pronto posible.

Al realizar las pruebas, se determinó que fue un fallo no tener en cuenta el nivel de conocimiento en informática, ya que  este nivel marca mucho la diferencia entre los usuarios a la hora de transmitir las opiniones sobre posibles mejoras.
Un 75\% de los usuarios que se escogieron fueron informáticos, y a la hora de devolver el feedback sus respuestas no eran de un usuario común, sino que devolvían las respuestas con una solución que implementaron ellos, aunque ni tan siquiera supieran programar con Android.
Eso fue un problema, ya que no daban casos de uso, sino que sólo encontraban fallos. Por este motivo se les indicó explícitamente  que dieran por lo menos 2 casos de uso nuevos que les parecieran interesantes incluir en una aplicación de este estilo.
De estas ideas salió el añadir las notas a las asignaturas o poder borrar y modificar los exámenes.

Para el segundo ciclo se tuvo que poner un filtro para el tipo de usuario que se buscaba conseguir, ya que en las primeras pruebas se consiguió ese dato que tuvo en cuenta para las siguientes y no se agregó tantos usuarios de la facultad de informática. A los que se les incluyó se les dijo especialmente que actuaran sin tener en mente la implementación.
La charla con los Baldusers informáticos surtió efecto sólo con la mitad de los nuevos usuarios de informática, ya que seguían hablando como si lo fueran a implementar ellos o con un lenguaje técnico que después de muchas frases lo único que pedían era cambiar los colores de un botón. 
Se optó por buscar en distintas facultades, amigos y gente cercana que pudieran probar la aplicación, el mayor problema era que cuanto más se abría el circulo de personas que probaban la aplicación, más difícil era seguirles la pista de lo que hacían. 
Se unió gente de facultades de ADE o Química, una propuesta interesante de mi tutor fue agregar a algún profesor, pero por falta de tiempo al tener que agregar y ayudar a los usuarios que se habían unido en este ciclo no se pudo contactar con ninguno.
Pero ya que no se podía disponer de un profesor, se decidió variar la edad de los Baldusers y comprobar cómo funcionaba la aplicación con ese cambio.

Cuando se decidió agregar a gente de diferente edad y que fuera cercana a mí, sin pensármelo dos veces recurrí a la familia, ellos no pueden decir que no.
El problema fue precisamente que al no poder decir que no, se unieron,  pero alguno no realizó el feedback o no pudieron instalársela por falta de tiempo o conocimiento.
Si hubieran sido los primeros usuarios, hacerles un seguimiento más cercano sí que hubiera sido posible, pero ya habiendo tanta gente metida y teniendo que hablar con cada uno por separado para no contaminar los pensamientos que tienen sobre la aplicación, tuve que optar por mandarles un manual de instalación realizado por mí, y en el caso que no funcionara, decirles que no se preocuparan.

Uno de los familiares era mi madre que tenía el perfil de una mujer adulta con conocimientos mínimos en informática, y que sabía manejar el móvil pero a un nivel usuario básico.
Y el otro familiar era mi primo, un adolescente de 16 años que está cursando bachiller, se decidió incluirle aunque fuera una aplicación dirigida especialmente para universitarios, para saber la acogida y el funcionamiento que se le podría dar fuera de la universidad y si se podría o no extrapolar a otras áreas como colegios o institutos.
Durante este ciclo, la parte más complicada fue incluir a la gente en el proyecto y conseguir que se descargaran la aplicación, ya que se había decidido no incluir a gente que tuviera mucho conocimiento de informática, eso aumentaba el tiempo de explicación a la hora de realizar la descarga.
Las fechas en las que se escogió a la gente no fueron acertadas ya que era principio de exámenes y los Baldusers estaban a mil cosas, y no podían realizar el feedback en las fechas señaladas.

A la hora de trabajar con usuarios hay que saber qué tipo de usuarios interesa tener en el proyecto, y cuáles aportaran más información.
Mucha de las cosas que digan los usuarios hay que filtrarlas y compararlas con su tipo de perfil a la hora de realizarlas, por ejemplo si sólo uno de los usuarios se ha quejado de la letra y justo ese usuario es el único que tiene un modelo de móvil de los más viejos posibles, igual no compensa realizar muchas modificaciones que puedan alterar el diseño de los otros usuarios que no se han quejado. Aparte esos usuarios que tienen en este momento un móvil viejo, llegará un punto en el que lo tengan que actualizar por uno más nuevo, y llegado ese punto se podrá prescindir de esa actualización de la letra.

Se ha hablado del tema de los usuarios como personas y de sus conocimientos, pero hasta el momento no se le ha dado importancia ni se ha hablado de algo muy importante que afecta a los usuarios, su dispositivo móvil. 
Cada usuario es distinto, aparte de lo mencionado antes, también por el tipo de dispositivo que usan, hay usuarios que tienen móviles con pantallas enormes que no les entra ni en el bolsillo y otros con una pantalla tan pequeña que lo pueden llevar en el bolsillo de la camisa.
Aparte de la pantalla, un aspecto importante es la versión del móvil, ya que no es lo mismo un móvil actual con la versión más nueva o un móvil de hace 3 años con una versión muy antigua de Android.
Por eso antes de escoger a los usuarios, hay que tener alguna idea del tipo de dispositivo que usan, por lo menos el tamaño de la pantalla o si disponen de teléfono, de poco serviría un usuario que no disponga de teléfono a menos que sólo se necesiten ideas a nivel conceptual.




\section{Pruebas}
\label{secc:pruebas}

A la hora de realizar las pruebas con los Baldusers se decidió seguir una metodología utilizada en la asignatura de Interacción Persona Computador, que consiste en ofrecer unas pautas a seguir al usuario, llevar la cuenta de las pulsaciones que realiza y el tiempo que tarda en realizar las acciones.
Para empezar a realizar las pruebas, un punto importante era conseguir que se descargaran la aplicación.
Fue una dura tarea, ya que Google no permite realizar invitaciones individuales a la aplicación, así que hubo que incluirlos en un grupo y después confiar en que ellos supieran aceptar la invitación y unirse a la fase alpha.

Una vez logrado el objetivo de instalar la aplicación en los dispositivos de los usuarios, la siguiente tarea fue la de pedirles información referente al tipo de usuario que eran, esta información se consiguió mediante unas preguntas realizadas por medio de formularios que rellenarían una vez, y con eso se tenía una idea del tipo de perfil que tenía el usuario.
Se les dio una semana para que pudieran realizar las pruebas pertinentes y acostumbrarse a la aplicación, durante esa semana se les pidió que fueran escribiendo el feedback que se les ocurriese en un documento compartido que se les había pasado por medio de la aplicación y de la comunidad de Google.
Algunos lo rellenaron sin poner problemas, otros hablaron directamente conmigo diciéndome el feedback y otros no realizaron ninguna de las dos cosas anteriores.

Para estos últimos usuarios, se tuvo que estar metiendo presión y haciéndoles acordar que se necesitaba el feedback lo antes posible. Ellos acusaron falta de tiempo y sólo dieron aportes estéticos, no un feedback consistente como se había preparado en un principio.
Si se querían realizar las modificaciones para la versión siguiente, se precisaba de los casos de uso que dijeran los usuarios, por este motivo por el cual no decían ningún caso de uso, se les mandó la tarea de que pensaran casos de uso útiles para la aplicación.
Los Baldusers respondieron positivamente, ya que no se les exigió esta vez escribir en ningún documento, sólo se les dijo que me lo comunicaran por el mejor medio posible.
Durante el segundo ciclo de pruebas, la cantidad de usuarios aumentó y con ello el tiempo que tenía que pasar ayudándoles también.
De lo aprendido en la vez pasada sobre los documentos de feedback con los usuarios, y para no estar esperando que escribieran, se les facilitó mi número de teléfono y mi correo electrónico, para que si se les ocurría algún caso de uso o les daba algún error me lo comunicaran por cualquiera de esos medios. Ya que había confianza con estos usuarios y al ser un grupo reducido, no importaba usar dichos medios de comunicación para leer a cada Balduser, ya que se asumía que si en la vez pasada que siendo 4 sólo la mitad cumplió correctamente, esta vez sólo tendría que hablar como mucho con 5 personas. Las demás personas o bien no me hablarían o bien si lo hacían serian unas pocas líneas.

Al recibir los mensajes de los Baldusers iba tomando notas de lo que pedía cada uno, y ya que no tenía que esperar a que me lo escribieran, podía empezar a darle prioridad y preguntarles directamente a qué se referían con aspectos de diseño, o cómo les parecía mejor si una opción u otra.
Esta solución funcionó mejor, no era tan profesional como recolectar la información directamente mediante formularios y documentos, pero al hacerlo de esta manera, los Baldusers no  se sentían tan obligados a realizar tareas y les daba menos pereza hacerlo, ademas siendo la mayoría estudiantes universitarios, estaban siempre con el Whatsapp o Telegram encendido, medio el cual usé para relacionarme con ellos lo más rápido posible al saber que ya habían probado la aplicación.

Cuando se producía un error en alguno de los dispositivos de algún Balduser, me llegaba al correo electrónico una notificación y todas las líneas de código que habían activado tal error.
De esta forma, el tiempo que me ahorraba al preguntar el motivo a cada Balduser se reducía enormemente.
Aprovechando ese tiempo en arreglar los fallos, o investigar cómo y por qué se producían.
Hubo situaciones en las que al no bloquearse la aplicación no se guardaba el fallo en Splunk Mint\cite{SplunkMint}, en esos casos sí que era beneficioso que el Balduser en cuestión se pudiera poner en contacto conmigo para decirme el motivo y cuándo se producía el fallo.

Al estar todo en funcionamiento y tener los casos de uso listos sin fallos, se propuso a los Baldusers que tuvieran tiempo quedar conmigo para realizar algunas pruebas.
Al ser fechas complicadas cuando se realizaron las pruebas, muchos no pudieron quedar conmigo en persona. A los que sí pudieron, se les realizó unas pruebas con la aplicación. Las pruebas consistían en lo siguiente:
Como primera prueba tenían que crear una asignatura, se les dejaba la aplicación abierta en el menú principal, y sin decirles nada se veía como reaccionaban a la tarea que se les había asignado. La tarea era una muy concreta, no tenían que pensar qué rellenar ni nada, todo lo que se podía realizar en la tarea estaba ya escrito en la pregunta.
Era interesante comprobar que dependiendo del tipo de Balduser que usaba la aplicación entendía los conceptos de manera distinta, y también la manera de interactuar con la aplicación era totalmente distinta entre unos y otros.

Hubo Baldusers que no estaban acostumbrados a realizar pulsaciones largas sobre objetos, y al no haber incluido en el menú dichas acciones se quedaban bloqueados en distintas preguntas.
Por este motivo se decidió modificar el diseño de la aplicación para que fuera más sencillo para el usuario.
También aparte de darles las tareas que tenían que realizar, se llevaba la cuenta de pulsaciones y el tiempo que tardaban en cada tarea.
Las pulsaciones y el tiempo se usó posteriormente para calificar las tareas de más complicadas a menos complicadas dependiendo del número de pulsaciones y del tiempo invertido en realizaras.

Al finalizar el ciclo se volvió a realizar las modificaciones en la aplicación pero ya realizando una aplicación consistente y sin errores, ya no se añadiría nuevos casos de uso y los casos nuevos de uso que dijeron los usuarios, aunque fueron pocos, se incluyeron en el apartado de posibles mejoras.
Una vez comprobado por el desarrollador que las modificaciones y los errores se solucionaban, se volvió a subir una nueva versión de la aplicación. En esta ocasión se agregó a un nuevo usuario que se había quedado fuera en la vez pasada, ya que estaba ocupado en esas fechas.
El usuario que se agrego era miembro de Magna SIS, ya que durante el proyecto me había puesto en contacto con la empresa para poder hacer uso de uno de sus proyectos ya realizados en el pasado, aunque al final no se pudo usar, se quiso que algún integrante de la empresa formara parte de los Baldusers.

La última fase de pruebas ya no era como las fases anteriores, en esta fase sobre todo se utilizaba para perfilar el producto y comprobar que no hubiera ningún cabo suelto sobre posibles errores o problemas de diseño no evaluados durante las versiones anteriores.
Gracias a esta fase de pruebas, se pudieron depurar fallos con algunas modificaciones y sobre todo con el caso de uso de Backup que se agregó como último caso de uso en la aplicación.
Para esta fase de pruebas, se usó una semana para que los Baldusers comprobaran los fallos que se producían con la nueva versión.
Se estuvo revisando cada día el Splunk Mint, comprobando que todo funcionara adecuadamente y se realizó la subida de la nueva versión.
A partir de este punto se repitió el proceso pero sólo para solucionar problemas, ya que no se tenía que unir a más usuarios, eso permitía que el tiempo que no se invertía en ello, se pudiera utilizar realizando los arreglos de forma más rápida.



\section{Tipos de documentos usados}
\label{secc:tipos de documentos usados}

Para la recogida de información por parte de los usuarios se empezó a usar documentos de Google Drive, compartiendo a cada Balduser el acceso y permisos de modificación.
Esos documentos eran un formulario donde el usuario respondía una serie de preguntas sobre sí mismo y sobre otras acerca de su dispositivo móvil. Aparte del formulario había documentos de texto que servían para distintos propósitos.
Uno de esos documentos era una explicación de por qué se había decidido realizar la aplicación y los usos que tenía en esa versión.
Otro documento era un manual donde se les explicaba a los Baldusers cómo poder instalarse la aplicación si se habían registrado por medio de Google plus, y ademas dos documentos donde los Baldusers escribían el feedback.
Se comprobó que el uso que daban los usuarios a esos documentos no era tan ágil como se pensaba, por ese motivo se decidió cambiar la forma de recoger esa información.
Se empezó a pedir a los nuevos usuarios la información por medio de e-mails o mensajes entre móviles. Yo personalmente los redactaba en los ficheros antes mencionados o tomaba apuntes de las modificaciones que se tenían que realizar.
En el caso que las modificaciones no estuvieran claras, el tener contacto directo con el usuario me permitía preguntarle el motivo de la modificación y entender por qué se había llegado a esa decisión.


\section{Medios de comunicación}
\label{secc:medios de comunicación}

Un punto importante y sumamente necesario a la hora de trabajar con los Baldusers fue le medios de comunicación que teníamos para intercambiar información.
Era muy importante escoger el medio de comunicación antes de empezar a agregar a alguien al proyecto, y tener bien claro el tipo de usuario con el que estaría hablando. No era lo mismo invitar a una persona mayor a probar la aplicación, que solo tiene conocimientos básico de uso del móvil, que a estudiantes de universidad que siempre llevan el móvil en las manos.
Por esos motivos se decidió hacer que el medio de comunicación fuera de la forma más ágil posible.

Al principio se le quiso dar un toque más profesional al realizar formularios y documentos donde los Baldusers tenían que realizar el feedback, pero no funcionó así, que se optó por el uso de los móviles como canal de comunicación.
Aparte de usar el móvil como canal de comunicación, por medio de Whatsapp o Telegram en algún caso, se siguió usando Google Drive para compartir el manual de instalación y los grupos de Google para enviar mensajes grupales.
Al primer grupo de Baldusers se les creó un grupo de Whatsapp para que compartieran por ahí sus dudas con respecto a cómo instalar la aplicación. Pero no resultó como se esperaba, al no conocerse entre ellos los usuarios no hablaba por el grupo y el grupo dejo de usarse pocos días después de crearse.
Por ese motivo, a los siguientes grupos que fueron entrando se les agregó a la comunidad de Google, y se les iba informando por ahí o directamente por correo electrónico o por Whatsapp.

Un buen método para intercambiar información con los usuarios,  es quedar con ellos en persona, aunque suponga más trabajo y pérdida de tiempo por tener que encontrar un rato en el que podamos quedar, de esta forma no hay dudas con lo que el usuario quiere.
Cosas como malentendidos o diferentes puntos de vista al ver una función se pueden solucionar mejor si se hablan las cosas a la cara que haciéndolo a través del teléfono.
Por este motivo, cuando era crucial saber lo que querían los usuarios, o para tomar los tiempos, se decidió quedar en persona en vez de decirles a los propios usuarios que se tomaran los tiempos de realización.



\section{Aportaciones Realizadas}
\label{secc:aportaciones Realizadas}

La mayoría de los Baldusers aportaron grandes ideas al proyecto, ya que hay muchas funcionalidades necesarias que si no hubieran sido por ellos, se hubieran pasado por alto. Por ejemplo la descripción en la parte de los exámenes no se había tenido en cuenta en un principio, pero después de que uno de los usuarios lo comentara, sí que se le dio utilidad.
En el tema de los errores, los Baldusers fueron de gran utilidad, ya que ellos fueron los que descubrieron la mayor parte de los errores dentro de la aplicación, al usar día a día surgen situaciones que no son habituales, como que te llamen por teléfono mientras está la aplicación abierta o situaciones parecidas, en esas situaciones uno no piensa cuando está implementando el código.
Los aportes que reportaba cada usuario se iban apuntando en un fichero de texto, y cuando concluía el periodo de prueba, se solucionaban los errores más grandes que se habían producido, ya que algunos de los casos de uso implementados que daban error puede que no fueran necesarios en la siguiente iteración.


\section{Usuarios y su relación con las aplicaciones móviles}
\label{secc:usuarios y su relación con las aplicaciones móviles}

Cada vez es más frecuente ver por la calle a todo el mundo con un móvil en las manos, cada persona está acostumbrada a usar las aplicaciones que más le gustan, pero hay aplicaciones que la sociedad las tiene como costumbre. Se han metido tanto en nuestro día a día que no se puede vivir sin ellas. Aplicaciones de mensajería instantánea como son Whatsapp o Telegram son comunes en cualquier dispositivo móvil, han sustituido a las llamadas y los mensajes por la facilidad, rapidez y bajo coste que les supone a los usuarios.
Además de eso, el acceso a Internet está al alcance de la mano de cualquiera mediante los navegadores móviles.

Google tiene un gran monopolio de las aplicaciones más usadas por los usuarios, todos los dispositivos tienen los servicios de Google instalados, lo que hace que por defecto ya se tenga preinstalado antes de usar el móvil, Youtube, Gmail, Calendar, Drive y muchos servicios más de Google.
Todas estas aplicaciones son populares por un motivo, porque sirven para que las personas nos relacionemos unas con otras, y Google lo ha conseguido sin lugar a dudas por medio de sus aplicaciones, no hay aplicación de Google que no tenga la función de compartir.
Aparte de esto, como es una familia de aplicaciones, cada una de estas está de alguna forma asociadas con sus hermanas, ya sea porque dan posibilidad de subir un fichero a esa aplicación o dan permiso para compartir a los usuarios de tu correo electrónico.

Los Baldusers que probaron la aplicación eran usuarios de todo este tipo de aplicaciones. Tenían los diseños de estas aplicaciones metidos en la cabeza y también sus funcionalidades, un problema para ser un desarrollador compitiendo contra por ejemplo la empresa de Google.
La mayoría pedían lo que habían visto en otros sitios y cómo el mercado está lleno de aplicaciones con mucha inversión y parece sencillo su uso, las proponen como casos de uso. 
El problema de ese tipo de situaciones es cómo decirles a los usuarios que no puedes desarrollar eso, ya que no tienes las herramientas o directamente te pueden denunciar por plagio.
Pero no lo entenderán y seguirán queriendo que se les haga un Calendar con otro nombre pero con las mismas funciones.


\section{Problemas encontrados con los usuarios}
\label{secc:problemas encontrados con los usuarios}

Uno de los problemas encontrados al trabajar con usuarios, es la cantidad de tiempo que se gasta al tener que hablar con ellos y explicarles cómo funciona cada cosa.
Lo habitual es que los usuarios pregunten dudas, el desarrollador se las resuelve y para no molestarle más, estos dicen que lo han entendido aunque no sea así. Al cabo de un tiempo volverán a preguntar la misma duda. Por este motivo, es más cómodo  realizar manuales de uso para las preguntas más comunes o para la fase de instalación.

Un tema muy importante si se va a trabajar con usuarios es definir los usuarios previamente, y qué tipo de respuestas se quiere recibir de ellos. Si se quieren respuestas útiles a la hora de desarrollar la aplicación, hay que escoger bien al tipo de usuario.
Los usuarios cuando devuelven un feedback suelen comparar con aplicaciones ya existentes.
Los usuarios informáticos hablaban como si fuera una aplicación web, los que no eran informáticos la comparaban con cualquier aplicación que han visto parecida, o directamente pedían que hiciera magia y convirtiese  una agenda de estudiantes en un secretario personal que les llevase los apuntes al día y les hiciese los trabajos.

En el apartado de la interacción de los usuarios con las aplicaciones móviles, ya se ha hablado de los problemas que nos encontramos al ser un desarrollador frente a grandes compañías, por eso mismo hay que tener claro desde un principio lo que se quiere realizar, y por mucho que se trabaje con usuarios la meta es clara, y el desarrollador es el que tiene que llevar el rumbo de la aplicación adelante. Es correcto que los usuarios pidan lo que han visto que hay por Internet, lo que han probado y les ha gustado. Pero aquello han probado y que probablemente utilizarían una o dos veces por semana, es una tarea que el desarrollador debe trabajar durante varios meses para conseguir funcionamiento parecido, sabiendo que no resultará igual un trabajo que ha realizado una única persona, que el realizado con un equipo de 100.


% Mediante los siguientes comandos, podrás compilar el conjunto de ficheros desde este mismo documento


%%% Local Variables: 
%%% mode: latex
%%% TeX-master: "../Principal"
%%% End: 


















% line in order to check if utf-8 is properly configured: áéíóúñ