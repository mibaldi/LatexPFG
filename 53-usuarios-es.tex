\chapter{Usuarios}
\label{ch:Usuarios}


% El siguiente es el texto que aparece en el encabezamiento del proyecto fin de grado (si se quiere que sea distinto al nombre del capítulo))
\chaptermark{PFG}

El método de trabajo usado en este proyecto se basa totalmente en los usuarios, como método de creación de una aplicación. Las aplicaciones tanto para móviles como para ordenador tienen un público definido. Es por ello que se les tiene que dar importancia.
Hay desarrolladores que el proceso de pruebas con usuarios lo realizan una vez ya se han creado los casos de uso y los necesita para buscar fallos en su producto. El método que se quiere usar en esta ocasión es un tanto distinto, también se les usara principalmente para facilitar al desarrollador a realizar todas las pruebas sobre la aplicación. Pero no es la finalidad principal para estos usuarios, el enfoque que se les ha dado a los usuarios esta vez ha sido un trabajo más en equipo con el desarrollador.
Los usuarios son los que van a usar la aplicación y los que van a tener que querer descargársela, si los casos de uso que se implementan no están dirigidos a todo tipo de usuarios, el número de personas que se la descarguen será pequeño.
Por este motivo los usuarios irán diciendo que casos de uso se incluirán en las siguiente versiones de la aplicación y como la construirían ellos si la fueran a usar día a día.
El método consiste en la implementación de la aplicación por ciclos de pequeños que no superen el mes de desarrollo.
Cada ciclo tendrá unos usuarios específicos, y la aplicación ira creciendo y modificándose al cabo de los ciclos. Para que al finalizar cada ciclo se tenga una aplicación funcional con más casos de uso que la anterior.
 
El funcionamiento es simple:
Para empezar el primer ciclo serán personas asociadas al proyecto en el caso de Baldugenda, el desarrollador y el tutor del proyecto.
En este ciclo se desarrollara la idea y se implementaran casos de uso simple pero funcional.
Cuando ya se consiga un producto más o menos enfocado a lo que quiere ser la idea final se incluyen 4 o 5 personas nuevas. Estas personas deben ser usuarios que el desarrollador sepa que van a probar su aplicación y se molestarán en responderle a las preguntas sobre la aplicación. 
En este ciclo el desarrollador se pondrá en contacto con esos usuarios y les hará llegar la aplicación. Ellos la probarán y le dirán al desarrollador lo que añadirían y modificarían de la versión inicial. El desarrollador aparte de escuchar las propuestas les realizara una serie de encuestas para comprobar que tipo de usuarios son.
Al finalizar el ciclo con los usuarios anteriores el desarrollador tendrá que escoger los casos de uso que se van a realizar de forma inmediata, los que se realizaran en versiones siguientes o los que se quedan fuera del proyecto.
Implementara los casos elegidos y da paso a un nuevo ciclo donde aumenta el número de usuarios.
En este ciclo el tipo de usuario es muy importante, no importa tanto la cantidad de personas sino el tipo de persona a la que se está metiendo para hacer las pruebas.
En este ciclo se duplicaran el número de usuarios que se había escogido para la anterior y se escogerán de 8 a 10 usuarios nuevos, con distintos tipos de perfil, y se subirá una versión nueva de la aplicación con los cambios realizados.
Habrá que avisar a los usuarios antiguos que se ha subido una versión nueva y a los nuevos darles acceso y decirles que se la descarguen.
Al finalizar este ciclo se tendrán casos de uso nuevos de 15 personas distintas que habrá que escoger cual implementar y que modificaciones hacer.
Se realizaría una versión nueva y se subiría y los usuarios invitados la testearían para encontrar fallos.


\section{Tipo de usuarios}
\label{secc:tipo de usuarios}

En Baldugenda se han estado usando diferente tipos de usuarios y se ha podido comprobar la diferencia de usar distintos perfiles de personas para desarrollar una aplicación móvil.
Hay distintos ámbitos que separan los perfiles de los usuarios, se pueden dividir por edades, también por tipos de estudios, por conocimientos sobre el tema, o por sexo.
Aparte de esas divisiones hay muchas otras pero en el caso de Baldugenda no se precisó de más para sacar conclusiones.
La mayoría de los usuarios que probaron la aplicación fueron hombres, hubo dos mujeres en la fase de usuarios y las diferencias de la división por sexo no se encontraron, así que no se le dieron importancia en las fases siguientes.
Hay colectivos como los discapacitados que son usuarios especiales a los que hay que desarrollar cosas muy específicas para ellos, por este motivo y porque si se realizaba un trabajo de este estilo hubiera aumentando muchísimo el coste se decidió no incluirlos en las divisiones y añadirlo dentro de las exclusiones del proyecto.
Dos divisiones que sí que se vieron muy afectadas a la hora de hacer las pruebas fueron la de la edad y los conocimientos sobre el tema, en esta situación se hablaban sobre el tema informático y manejo de dispositivos móviles.

 
Se tuvo que escoger el tipo de usuario que se buscaba que fuera significativo dentro de la aplicación, por ello y viendo el tipo de aplicación que se quería realizar se decidió que el usuario final debía ser usuarios universitarios, ya que la aplicación tenía como puntos muy específicos las asignaturas y los exámenes, asuntos que van muy ligados al ámbito de los estudiantes.
Esa decisión marco el tipo de usuario que se escogería en las primeras pruebas, las 4 personas primeras eran universitarios todos y no se le dio importancia a sus conocimientos sobre informática ni sobre móviles, lo único que importo fue que tuvieran dispositivos compatibles y que tuviera relación cercana a ellos para poder tener los resultados lo más pronto posible.
Al realizar las pruebas se determinó que fue un fallo no tener él cuenta el nivel de conocimiento en informática, el nivel de conocimientos en informática marca mucho la diferencia entre los usuarios a la hora de preguntar opiniones.
Un 75\% de los usuarios que se escogieron fueron informáticos y a la hora de devolver el feedback sus respuestas no eran de un usuario común sino que devolvían las respuestas con una solución que implementarían ellos, aunque ni tan siquiera supieran programar con Android.
Eso fue un problema, ya que no daban casos de uso sino que solo decían de arreglar cosas. Por este motivo se les indico explícitamente  que dieran por lo menos 2 casos de uso nuevo que les parecieran interesante incluir en una aplicación de este estilo.
De estas ideas salió el añadir las notas a las asignaturas o poder borrar y modificar los exámenes.
Para el segundo ciclo se tuvo que poner un filtro para el tipo de usuario que se buscaba conseguir, ya que en las primeras pruebas se consiguió el dato ese, se tuvo en cuenta para las siguientes y no se agregó tantos usuarios de la facultad de informática y a los que se les incluyo se les dijo especialmente que actuaran sin tener en mente la implementación.
La charla con los Baldusers informáticos surtió efecto solo con la mitad de los nuevos usuarios de informática. Seguían hablando como si lo fueran a implementar ellos o con un lenguaje técnico que después de muchas frases lo único que pedían era cambiar los colores de un botón. 
Se optó por buscar en distintas facultades, amigos y gente cercana que pudieran probar la aplicación, el mayor problema era que cuanto más se abría el circulo de personas que probaban la aplicación, más difícil era seguirles la pista de lo que hacían. 
Se unió gente de facultades como ADE o Química, una propuesta interesante de mi tutor fue agregar a algún profesor, pero por falta de tiempo al tener que agregar y ayudar a los usuarios que se habían unido en este ciclo no se pudo contactar con ninguno.
Pero ya que no se podía disponer de un profesor, se decidió variar la edad de los Baldusers y comprobar que tal funcionaba la aplicación con ese cambio.



Cuando se decidió agregar a gente de diferente edad y que fuera cercana a mí, sin pensármelo dos veces recurrí a la familia, ellos no pueden decir que no.
El problema fue precisamente ese que al no poder decir que no se unieron pero alguno no realizaron el feedback o no pudieron instalársela por falta de tiempo o conocimiento.
Si hubieran sido los primeros usuarios, hacerles un seguimiento más cercano sí que hubiera sido posible, pero ya habiendo tanta gente metida y teniendo que hablar con cada uno por separado para no contaminar los pensamientos que tienen sobre la aplicación tuve que optar por mandarles un manual de instalación hecho por mí, y en el caso que no funcionara decirles que no se preocuparan.
Uno de los familiares era mi madre, tenía el perfil de una mujer adulta con conocimientos mínimos en informática y que sabía manejar el móvil pero a un nivel usuario básico.
Y el otro familiar era mi primo, un adolescente de 16 años que está cursando bachiller, se decidió incluirle aunque fuera una aplicación dirigida especialmente para universitarios, para saber la acogida y el funcionamiento que se le podría dar fuera de la universidad y si se podría o no extrapolar a otras áreas como colegios o institutos.
Durante este ciclo la parte más complicada fue incluir a la gente en el proyecto y conseguir que sé descargaran la aplicación, ya que se había decidido no incluir a gente que tuviera mucho conocimiento de informática, eso aumentaba el tiempo de explicación a la hora de realizar la descarga.
Las fechas en las que se escogio a la gente tampoco no fue acertada ya que era principio de exámenes y los Baldusers estaban a mil cosas y no podían realizar el feedback en las fechas señaladas.
 
A la hora de trabajar con usuarios hay que saber qué tipo de usuarios interesa tener en el proyecto y cuales aportaran más información.
Mucha de las cosas que digan los usuarios hay que filtrarlas y compararlas con su tipo de perfil a la hora de realizarlas, por ejemplo si solo uno de los usuarios se ha quejado de la letra y justo ese usuario es el único que tiene un modelo de móvil de los más viejos posibles, igual no compensa realizar muchas modificaciones que puedan alterar el diseño de los otros usuarios que no se han quejado. Aparte esos usuarios que tienen en este momento es móvil viejo llegara un punto en el que lo tengan que actualizar por uno más nuevo y llegado ese punto se podrá prescindir de esa actualización de la letra.
Se ha hablado del tema de los usuarios como personas y de sus conocimientos, pero hasta el momento no se le ha dado importancia ni se ha hablado de algo muy importante que afecta a los usuarios, su dispositivo móvil. 
Cada usuario es distinto aparte de lo mencionado antes, también por el tipo de dispositivo que usan, hay usuarios que tienen móviles con pantallas enormes que no les entra ni en el bolsillo y otros con una pantalla tan pequeña que lo pueden llevar en el bolsillo de la camisa.
Aparte de la pantalla un aspecto importante es la versión del móvil, no es lo mismo un móvil actual con la versión más nueva o un móvil de hace 3 años con una versión muy antigua de Android.
Por eso antes de escoger a los usuarios hay que tener alguna idea del tipo de dispositivo que usan, por lo menos el tamaño de la pantalla o si disponen de teléfono, de poco serviría un usuario que no disponga de teléfono a menos que solo se necesiten ideas a nivel conceptual.



\section{Pruebas}
\label{secc:pruebas}

A la hora de realizar las pruebas con los Baldusers se decidió seguir una metodología aprendida en la asignatura de Interacción Persona Computador, que es la de darle unas pautas a seguir al usuario y llevar la cuenta de las pulsaciones que realiza y el tiempo que tarda en realizar las acciones.
Para empezar a realizar las pruebas un punto importante era conseguir que se descargaran la aplicación.
Fue una tarea dura, ya que google no permite realizar invitaciones individuales a la aplicación así que había que incluirlos en un grupo y después confiar en que ellos supieran aceptar la invitación e unirse a la fase alpha.
Una vez logrado el objetivo de instalar la aplicación en los dispositivos de los usuarios, la siguiente tarea fue la de pedirles información referente al tipo de usuario que eran, esta información se consiguió mediante unas preguntas realizadas por medio de formularios que rellenarían una vez y con eso se tenía una idea del tipo de perfil que es el usuario. 
Se les dio una semana para que pudieran realizar las pruebas pertinentes y acostumbrarse a la aplicación, durante esa semana se les pidió que fueran escribiendo el feedback que se les ocurriese en un documento compartido que se les había pasado por medio de la aplicación y de la comunidad de google.
Algunos lo rellenaron sin poner problemas, otros hablaron directamente conmigo diciéndome el feedback y otros no realizaron ninguna de las dos cosas anteriores.
Para estos últimos usuarios se tuvo que estar metiendo presión y haciéndoles acordar que se necesitaba el feedback lo antes posible. Ellos solo dieron largas y aportes estéticos, no un feedback consistente como se había preparado en un principio.
Si se quería realizar las modificaciones para la versión siguiente se precisaba de los casos de uso que dijeran los usuarios, por este motivo ya que no decían ningún caso de uso se les mando la tarea de que pensaran casos de uso útiles para la aplicación.
Los Baldusers respondieron positivamente, ya que no se les exigió esta vez escribir en ningún documento, solo se les dijo que me lo comunicaran por el medio que mejor les viniera.
 
Durante el segundo ciclo de pruebas la cantidad de usuarios aumento y con ello el tiempo que tenía que pasar ayudándoles también.
De lo aprendido en la vez pasada sobre los documentos de feedback con los usuarios y para no estar esperando que escribieran, se les facilito mi número de teléfono y mi correo, para que si se les ocurría algún caso de uso o les daba algún error me lo comunicaran por cualquiera de esos medios. Ya que había confianza con estos usuarios y ser un grupo reducido no importaba usar estos medios de comunicación para leer a cada Balduser ya que se asumía que si en la vez pasada que siendo 4 solo la mitad cumplió correctamente, esta vez solo tendría que hablar como mucho con 5 personas las demás o no me hablarían o si lo hacían serian unas pocas líneas.

Al recibir los mensajes de los Baldusers iba tomando notas de lo que pedía cada uno, y ya que no tenía que esperar a que me lo escribieran podía empezar a darle prioridad y preguntarles directamente en que se referían con aspectos de diseño o como les parecía mejor si una opción u otra.
Esta solución funciono mejor, no era tan profesional como recolectar la información directamente mediante formularios y documentos, pero al hacerlo de esta manera los Baldusers no  se sentían tan obligados a realizar cosas y les daba menos pereza hacerlo, aparte al ser la mayoría universitarios estaban siempre con el Whatsapp o Telegram encendido, medio que usé para relacionarme con ellos lo más rápido posible al saber que ya habían probado la aplicación.
Cuando se producía un error en alguno de los dispositivos de algún Balduser me llegaba al correo electrónico una notificación y todas las líneas de código que habían activado tal error.
De esta forma el tiempo que me ahorraba al preguntar el motivo a cada Balduser se reducía enormemente.
Aprovechando ese tiempo en arreglar los fallos, o investigar cómo y porque se producían.
Hubo situaciones en las que al no bloquearse la aplicación no se guardaba el fallo en Splunk Mint, en esos casos sí que era beneficioso que el Balduser en cuestión se pudiera poner en contacto conmigo para decirme el motivo y cuando se producía el fallo.
 
Al tener todo funcionando, los casos de uso listos sin fallos, se les propuso a los Baldusers que tuvieran tiempo, a quedar conmigo para realizar algunas pruebas.
Al ser fechas malas cuando se realizaron las pruebas, no muchos pudieron quedar conmigo en persona. A los que sí que pudieron, se les realizo unas pruebas con la aplicación, las pruebas consistían en lo siguiente:
Como primera prueba tenían que crear una asignatura, se les dejaba la aplicación abierta en el menú principal y sin decirles nada se veía como reaccionaban a la tarea que se les había asignado. La tarea era una muy concreta no tenían que pensar que rellenar ni nada, todo lo que se podía realizar en la tarea estaba ya escrito en la pregunta.
Era interesante comprobar que dependiendo del tipo de Balduser que usaba la aplicación entendía los conceptos de manera distinta. Y también la manera de interactuar con la aplicación era totalmente distinta entre unos y otros.
Hubo Baldusers que no estaban acostumbrados a realizar pulsaciones largas sobre objetos y al no haber incluido en el menú dichas acciones se quedaban bloqueados en distintas preguntas.
Por este motivo se decidió modificar el diseño de la aplicación para que fuera más sencillo para el usuario.
También aparte de darles las tareas que tenían que realizar, se llevaba la cuenta de pulsaciones y el tiempo que tardaban en cada tarea.
Las pulsaciones y el tiempo se usó posteriormente para calificar las tareas de más complicadas a menos complicadas dependiendo del número de pulsaciones y del tiempo invertido en realizaras.
Al finalizar el ciclo se volvió a realizar las modificaciones en la aplicación pero ya realizando una aplicación consistente y sin errores, ya no se añadiría nuevos casos de uso y los casos nuevos de uso que dijeron los usuarios, aunque fueron pocos, se incluyeron en el apartado de posibles mejoras.
Una vez comprobado por el desarrollador que las modificaciones y los errores se solucionaban se volvió a subir una nueva versión de la aplicación. En esta ocasión se agregó a un nuevo usuario que se había quedado fuera en la vez pasada, ya que estaba ocupado en esas fechas.
El usuario que se agrego era miembro de Magna SIS ya que durante el proyecto me había puesto en contacto con la empresa para poder hacer uso de uno de sus proyectos ya realizados en el pasado, aunque al final no se pudo usar, se quiso que algún integrante de la empresa formara parte de los Baldusers.
 
La fase de pruebas ultima ya no era como las fases anteriores, en esta fase sobre todo se utilizaba para perfilar el producto y comprobar que no hubiera ningún cabo suelto sobre posibles errores o problemas de diseño no testeados durante las versiones anteriores.
Gracias a esta fase de pruebas se pudieron depurar fallos con algunas modificaciones y sobre todo con el caso de uso de Backup que se agregó como último caso de uso en la aplicación.
Para esta fase de pruebas se usó una semana para que los Baldusers comprobaran los fallos que se producían con la nueva versión.
Se estuvo revisando cada día el Splunk Mint, comprobando que todo funcionara adecuadamente y se realizó la subida de la nueva versión.
A partir de este punto se repitió el proceso pero solo para solucionar problemas y ya que no se tenía que unir a más usuarios, eso permitía que el tiempo que no se invertía en eso, se pudiera utilizar realizando los arreglos más rápido.


\section{Tipos de documentos usados}
\label{secc:tipos de documentos usados}

Para la recogida de información por parte de los usuarios se empezó a usar documentos de Google Drive, compartiendo a cada Balduser el acceso y permisos de modificación.
Esos documentos eran un formulario donde el usuario respondía una serie de preguntas sobre sí mismo y otras sobre su dispositivo móvil. Y aparte del formulario había documentos de texto que servían para distintos propósitos.
Uno de esos documentos era una explicación de porqué se había decidido realizar la aplicación y los usos que tenía en esa versión.
Otro documento era un manual donde se les explicaba a los Baldusers como poder instalarse la aplicación si se habían registrado por medio de Google plus.
Y 2 documentos donde los Baldusers escribían el feedback.
Se comprobó que el uso que daban los usuarios a esos documentos no era lo ágil que se esperaba, por ese motivo se decidió cambiar la forma de recoger esa información.
Se empezó a pedir a los nuevos usuarios la información por medio de emails o mensajes entre móviles. Y yo personalmente los redactaba en los ficheros antes mencionados o tomaba apuntes de las modificaciones que se tenían que realizar.
En el caso que las modificaciones no estuvieran claras el tener contacto directo con el usuario me permitía preguntarle el motivo de la modificación y entender porque se había llegado a esa decisión.

\section{Medios de comunicación}
\label{secc:medios de comunicación}

Un punto importante y sumamente necesario a la hora de trabajar con los Baldusers fue le medios de comunicación que teníamos para intercambiar información.
Era muy importante escoger el medio de comunicación antes de empezar a agregar a nadie al proyecto y tener bien claro el tipo de usuario con el que estaría hablando, no era lo mismo invitar a mi abuelo a probar la aplicación, que justo sabe usar el móvil para llamar que a estudiantes de universidad que siempre llevan el móvil en las manos.
Por esos motivos se decidió hacer que el medio de comunicación fuera de la forma más ágil posible.
Al principio se le quiso dar un toque más profesional al realizar formularios y documentos donde los Baldusers tenían que realizar el feedback, pero no funciono así que se optó por el uso de los móviles como canal de comunicación.
Aparte de usar el móvil como canal de comunicación por medio de Whatsapp o Telegram en algún caso, se siguió usando Google drive para compartir el manual de instalación y los grupos de google para enviar mensajes grupales.
Al primer grupo de Baldusers se les creo un grupo de Whatsapp para que compartieran por ahí sus dudas con respecto a cómo instalar la aplicación. Pero no resulto como se esperaba, al no conocerse la gente no hablaba por el grupo y el grupo murió pocos días después de crearse.
Por ese motivo a los siguientes grupos que fueron entrando se les agregó a la comunidad de Google y se les iba informando por ahí o directamente por correo o por Whatsapp.
Un buen método para intercambiar información con los usuarios es quedar con ellos en persona, aunque suponga más trabajo y pérdida de tiempo por tener que encontrar un rato que los dos podamos quedar, de esta forma no hay dudas con lo que el usuario quiere.
Cosas como malentendidos o diferentes puntos de vista al ver una función se pueden solucionar mejor si se hablan las cosas a la cara que haciéndolo a través del teléfono.
Por este motivo cuando era crucial saber lo que querían los usuarios o para tomar los tiempos se decidió quedar en vez de decirles a los propios usuarios que se tomaran los tiempos de realización.


\section{Aportaciones Realizadas}
\label{secc:aportaciones Realizadas}

La mayoría de los Baldusers aportaron grandes ideas al proyecto, hay muchas funcionalidades que si no hubiera sido por ellos se hubiera pasado por alto que eran necesarias, por ejemplo la descripción en la parte de los exámenes, no se había tenido en cuenta en un principio, pero después de que uno de los usuarios lo comentara, sí que se le vio utilidad.
En el tema de los errores los Baldusers fueron de gran utilidad, ya que ellos fueron los que descubrieron la mayor parte de los errores dentro de la aplicación, al usar día a día surgen situaciones que no son habituales, como que te llamen por teléfono mientras esta la aplicación abierta o situaciones parecidas, en esas situaciones uno no piensa cuando está implementando el código.
Los aportes que reportaba cada usuario se iba apuntando en un fichero de texto y cuando concluía el periodo de prueba se solucionaban los errores más grandes que se habían producido, ya que alguna de las cosas que daban error puede que en la siguiente interacción no hiciera falta.

\section{Usuarios y su relación con las aplicaciones móviles}
\label{secc:usuarios y su relación con las aplicaciones móviles}

Cada vez es más frecuente ver por la calle a todo el mundo con un móvil en las manos, cada persona está acostumbrada a usar las aplicaciones que más le gustan, pero hay aplicaciones que la sociedad las tiene como costumbre. Se han metido tanto en nuestro día a día que no se puede vivir sin ellas
Una de estas aplicaciones es para la comunicación, quien diría que hace unos pocos años atrás se mandaban SMS y se hablaba con una persona para saber que tal estaba por teléfono, ahora todo eso se ha sustituido por Whatsapp y Telegram.
Además de eso el acceso a internet está al alcance de la mano de cualquiera mediante los navegadores móviles.
Google tiene un gran monopolio de las aplicaciones más usadas por los usuarios, todos los dispositivos tienen los servicios de google instalados lo que hace que por defecto ya se tenga preinstalado antes de usar el móvil, Youtube, Gmail, Calendar, Drive y muchos servicios más de Google.
El porqué  todas estas aplicaciones funcionan, es porque sirven para que las personas nos relacionemos unas con otras, y Google lo ha conseguido sin lugar a dudas por medio de sus aplicaciones, no hay aplicación de google que no tenga la función de compartir.
Aparte de esto como es una familia de aplicaciones cada una están de alguna forma asociadas con sus hermanas, ya sea porque dan posibilidad de subir un fichero a esa aplicación o da permiso para compartir a los usuarios de tu correo electrónico.
Los usuarios de Baldugenda no eran distintos de las demás personas.
Tenían sus ideas metidas en la cabeza, un problema para ser un desarrollador compitiendo contra por ejemplo la empresa de Google.
La mayoría piden lo que han visto en otros sitios y como el mercado está lleno de aplicaciones con mucha inversión y hasta parece sencillo y bonito al usarlas las proponen como casos de uso. 
El problema de ese tipo de situaciones es como decirles a los usuarios que no puedes desarrollar eso ya que no tienes las herramientas o directamente te pueden denunciar por plagio.
Pero no lo entenderán y seguirán queriendo que se les haga un Calendar con otro nombre pero las mismas funciones.

\section{Problemas encontrados con los usuarios}
\label{secc:problemas encontrados con los usuarios}

Uno de los problemas encontrados con trabajar con usuarios es la cantidad de tiempo que se gasta al tener que hablar con ellos y explicarles cómo funciona cada cosa.
Aparte una vez se les explica dicen que lo han entendido y después es mentira y dos días después te lo volverán a preguntar.
Un tema muy importante si se va a trabajar con usuarios es definir los usuarios antes y que tipo de respuestas se quiere recibir. Si se quiere respuestas de usuario normal hay que escoger muy bien, ya que parece que todo el mundo sabe de todo y comentan como si ellos hicieran todos los días una aplicación, el problema es el uso que tienen los usuarios con las aplicaciones habituales.
Los usuarios informáticos hablaran como si fuera una aplicación web, los que no sean informáticos te la comparan con cualquier aplicación que han visto parecida o directamente te pedirán que hagas magia y conviertas una agenda de estudiantes en un secretario personal que te lleve los apuntes al día y encima te haga los trabajos.
En el apartado de la interacción de los usuarios con las aplicaciones móviles ya se ha hablado de los problemas que nos encontramos al ser un pez en un mar de tiburones, por eso mismo hay que tener claro desde un principio lo que se quiere realizar y por mucho que se trabaje con usuarios la meta es clara y el desarrollador es el que tiene que llevar el rumbo de la aplicación adelante. Es correcto que los usuarios pidan lo que han visto que hay por internet, lo que han probado y les ha gustado, pero igual lo que ellos han probado y que lo usaran igual 1 o 2 veces a la semana, al final el que lo tiene que desarrollar tiene que invertir meses en conseguir un trabajo parecido y ya teniendo en la cabeza que no resultara igual un trabajo que ha realizado una persona frente al que ha realizado un equipo de 100 con gente de distintos ámbitos.

% Mediante los siguientes comandos, podrás compilar el conjunto de ficheros desde este mismo documento


%%% Local Variables: 
%%% mode: latex
%%% TeX-master: "../Principal"
%%% End: 


















% line in order to check if utf-8 is properly configured: áéíóúñ