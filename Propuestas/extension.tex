\section{Propuestas de extensión}

Las propuestas citadas a continuación son ideas aportadas por los Baldusers e ideas de origen propio, para que sirvan como propuesta para dar continuidad al trabajo presentado en esta memoria. 
\begin{itemize}
	\item Ya que Baldugenda está dirigido a los estudiantes universitarios, añadir dentro de los destinatarios de la aplicación a los profesores para que puedan llevar al día los exámenes y compartirlos con los estudiantes.
	\item Durante la fase de pruebas se comprobó que el funcionamiento de Baldugenda funcionaría correctamente en los institutos, realizando modificaciones como puede ser agregar trabajos y deberes.
	\item La ampliación de las funcionalidades de Baldugenda haciéndolas accesibles a las personas con discapacidad, sería una buena forma de conseguir usuarios que la mayoría de aplicaciones dejan de lado.
	\item La realización de Widgets para el móvil de la aplicación, donde se podría visualizar los exámenes y las asignaturas sin necesidad de abrir la aplicación, y de esta manera poder realizar notificaciones al usuario sin la necesidad de usar las de Google Calendar.
	\item La implementación de Baldugenda en distintos dispositivos Android, como SmartTv o Smartwatch.
	\item Migración de Baldugenda a otros sistemas operativos, siendo el prioritario IOs.
	\item Implementación de version web de Baldugenda.
	\item Sincronización de Baldugenda con Gaur para acceder a las asignaturas.
\end{itemize}