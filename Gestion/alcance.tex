\section{Gestión del alcance}
\label{secc:Gestión del alcance}

Ya se ha hablado en capítulos anteriores sobre el alcance y cómo ha ido variando dependiendo del ciclo en el que se estuviera. Este aumento del alcance ha sido gracias a los Baldusers que han ido aportando funcionalidades nuevas.
A continuación se enumerará el alcance del primer ciclo:
\begin{itemize}
	\item Creación de asignaturas indicando nombre, posibilidad de añadir enlaces y seleccionando el sistema de evaluación que se seguirá en esa asignatura. 
	\item Creación de exámenes indicando nombre, asignatura, calendario de Google Calendar, fecha y hora.
	\item Manejo básicos de calendarios de la cuenta de Google Calendar del usuario. 
	\item Los usuarios podrán escoger si quieren que se active la notificación en el examen creado dentro de Google Calendar.
	\item Borrado de exámenes desde el examen escogido.
	\item Visualizador de asignatura.
	\item Visualizador de examen.
	\item Lista de asignaturas y buscador dentro de la lista.
	\item Lista de exámenes, agrupados en asignaturas.
	\item Actividad para contactar con el desarrollador mediante email o teléfono.
	\item Uso del API de Google Calendar.
	\item Errores encontrados durante el desarrollo del prototipo resueltos.
\end{itemize}
\newpage
Una vez concluido el primer ciclo se realizó la invitación a los Baldusers y realizaron el feedback. Con el feedback devuelto se aumentó el alcance con las siguientes características:
\begin{itemize}
	\item Modificación del sistema de evaluación, enlaces y nota de una asignatura.
	\item Modificación de fecha,hora,asignatura,calendario y nota de un examen.
	\item Borrado de asignaturas que no tengan exámenes.
	\item Nuevo campo de nota en examen.
	\item Nuevo campo de nota en asignatura.
	\item Cambio de diseño de menú principal.
	\item Errores encontrados por los Baldusers resueltos.
\end{itemize}
Después del segundo ciclo, y después de añadir más Baldusers y recibir el feedback se aumentó de nuevo el alcance:
\begin{itemize}
	\item Nuevo campo de descripción en examen.
	\item Editar examen dentro de la actividad examen y desde la lista de exámenes.
	\item Editar asignatura dentro de la actividad asignatura y desde la lista de asignaturas.
	\item Borrar examen dentro de la actividad examen.
	\item Borrar asignatura dentro de la actividad asignatura y desde la lista de asignaturas.
	\item Exportar Base de datos de Baldugenda a la cuenta de Google Drive del usuario.
	\item Importar Base de datos de Baldugenda desde la cuenta de Google Drive del usuario.
	\item Uso del API de Google Drive.
	\item Errores encontrados por los Baldusers resueltos.
\end{itemize}

