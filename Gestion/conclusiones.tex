\section{Conclusiones sobre la gestión}
\label{secc:Conclusiones sobre la gestión}

Para finalizar el capítulo sobre la gestión del proyecto realizaremos un breve repaso de los apartados anteriores. Dentro de la gestión del alcance se puede observar que durante el primer ciclo se tuvo un mayor peso en las tareas de implementación y aprendizaje sobre las APIs de Google, por este motivo es el ciclo que más tiempo ha llevado dentro del proyecto. En los ciclos posteriores el tiempo invertido se dividió en la interacción con los Baldusers y las modificaciones que iban pidiendo que se realizaran en la aplicación.

En el apartado de costes se han dividido las categorías de la aplicación de la de los usuarios. Aunque están muy relacionadas, se invirtió un tiempo importante del proyecto en la realización de las pruebas tanto individuales como con usuarios, por este motivo esa tarea es la única en la que también están incluidas las pruebas con usuarios.
Aunque al principio se decidió trabajar con otro procesador de textos, al final acabamos decidiendo trabajar con Latex, ya que todos los aspectos de diseño y estructura serían más sencillos de realizar mediante la plantilla y se llevaría menos tiempo realizarlo.
Un aspecto muy importante para haber llevado correctamente el tiempo de cada una de las tareas, ha sido las anotaciones periódicas de cada una mediante los resúmenes de las reuniones y las subidas al Github, donde se subía una versión estable del proyecto cada 5 horas aproximadamente de trabajo.

El trabajo en ciclos fue muy útil de la forma en la que está pensado el desarrollo del proyecto. Al trabajar con usuarios, los ciclos se alargaban más que si se hubiera realizado sin el feedback de estos, aunque el tiempo que se tuvo que esperar a las respuestas de los usuarios se invirtió en probar más a fondo la aplicación, solucionar errores y poder realizar las entrevistas de una forma mas amena y calmada.
El tener separado la planificación de cada ciclo ha resultado muy útil a la hora de realizar una planificación general del proyecto, teniendo los hitos más importantes marcados, no ha habido ningún problema a la hora de hacer un cronograma.




