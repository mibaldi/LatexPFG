\chapter{Introducción}

Durante los últimos años el uso de \glspl{dispositivo movil} a ido en aumento, es difícil ver por la calle gente que no este usando un móvil o que no lo tenga en el bolsillo. A llegado hasta tal punto la necesidad de usar el móvil que las acciones que se realizaban sin la necesidad de esté se han visto transformadas a acciones virtuales en las que el usuario por medio de pulsaciones consigue el mismo resultado. Véase la compra por Internet o el envió de mensajes.
Los desarrolladores aprovecharon este tirón para llevar el mundo real a los dispositivos y con ello ganarse el mercado que antes se llevaban por ejemplo los supermercados, correos gimnasios, etc...
Viendo a un publico cercano a mi como son los estudiantes de universidad y viendo los problemas que les suelen suceder al cabo de los cursos universitarios se pensó en dar solución a estos problemas mediante Baldugenda.

En estos momentos la saturación de apps en el mercado de los móviles hace difícil encontrar problemas que no se hayan resuelto ya, pero hay ocasiones como la que se da en el caso de Baldugenda que se quiere buscar un publico especifico con unas funcionalidades muy concretas. Baldugenda se creó por la situación que se veía en los estudiantes universitarios, durante una carrera universitaria se llevan a cabo muchas asignaturas y dentro de cada asignatura pueden darse muchos exámenes con fechas que se mezclan, con materia que no se sabe si entra o no. En la sociedad antes de la llegada de los móviles la gente apuntaba las cosas en agendas o libretas para no olvidarse, este tipo de técnicas suponen el siguiente problema, si se quiere recordar la fecha de algo que estaba apuntado en la agenda se tiene que llevar encima siempre, o por el contrario si se quiere anotar algún evento nuevo se necesita disponer de la agenda y de un bolígrafo. Por estas razones hay alumnos que no usan agenda por que les da pereza sacar el bolígrafo cuando ya han guardado todo y se fían de su memoria. Pero hay algo que siempre todo universitario lleva encima con él, su móvil. Llevar la agenda en el móvil solucionaría el problema de bolígrafos o libretas. 

Lo que le hace diferente a Baldugenda de las otras aplicaciones aparte de su publico ya definido es la interacción mediante Google y sus aplicaciones.

En el trabajo realizado en este Proyecto de Fin de Grado ha tenido como objetivo la concepción, diseño e implementación de la aplicación que pretende dar solución a esta necesidad tan simple como es apuntar los exámenes en una hoja de papel.

Se ha desarrollado la aplicación de manera cíclica y las decisiones de implementación y diseño han sido realizadas por un grupo de usuarios llamados Baldusers. La integración de estos usuarios al proyecto ha repercutido en la realización de tareas de gestión exclusivas, con entrevistas guiadas con cuestionarios, comunicación constante y tecnologías para la automatización del feedback en caso de error.

\gls{Baldugenda} ha sido implementado como una app para Android, se ha descartado el desarrollo multiplataforma debido a las limitaciones, tras lo cual se ha decidido usar el sistema operativo Android por el conocimiento previo y mayor popularidad respecto a otras plataformas.

Una parte importante de Baldugenda son los servicios de Google, tanto Google Calendar, Google Drive como Google+ se han usado durante el desarrollo del proyecto y dan mayores funcionalidades a la aplicación.

Este documento, que es la memoria del trabajo realizado por Mikel Balduciel Diaz
bajo la dirección del doctor José Miguel Blanco Arbe durante el curso académico 2014/2015, está estructurado de la siguiente forma:

\begin{itemize}
	\item Los \textbf{objetivos} del proyecto, divididos en los siguientes apartados, \textbf{antecedentes}, el \textbf{alcance} y las \textbf{exclusiones}.
	\item La explicación del modelo seguido durante el \textbf{ciclo de vida del proyecto} y la \textbf{participación de los usuarios}.
	\item Se describe todo lo referido a los  \glspl{Balduser} dentro del apartado de \textbf{usuarios}, tanto el \textbf{tipo de usuario}, las \textbf{pruebas} realizadas, tipos de \textbf{documentos usados}, \textbf{formas de comunicación},\textbf{aportaciones},la \textbf{relación con las aplicaciones móviles} y un apartado de \textbf{problemas encontrados}.
	\item Para el cliente desarrollado en Android, Baldugenda se describe en el apartado de \textbf{Aplicación} tanto el \textbf{análisis y diseño}, como su \textbf{desarrollo y pruebas}.
	\item Un capitulo sobre el \textbf{desarrollo de apps de la categoría de Baldugenda}, se han separado en la parte de \textbf{Google} y las características de \textbf{Android}.
	\item El siguiente capítulo relata la \textbf{gestión del proyecto} en las áreas principales no tratadas en el resto de la memoria: el alcance, el tiempo y los costes y conclusiones sobre la gestión.
	\item El proyecto concluye con las \textbf{conclusiones} personales, como aspectos aprendidos al cabo del proyecto.
\end{itemize}

A lo largo de toda la memoria se repiten numerosos términos y acrónimos que podrían
ser desconocidos para el lector o tener, en el contexto de este proyecto, un significado distinto. Por ello, se adjunta un glosario que describe la definición de estas palabras, al cual conviene acudir en caso de duda.






































 %line in order to check if utf-8 is properly configured: áéíóúñ
