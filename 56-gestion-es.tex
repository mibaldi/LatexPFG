\chapter{Gestión del proyecto}
En el proyecto de Baldugenda se ha seguido de Lean Startup para desarrollar la aplicación, primero se ha desarrollado un producto mínimo viable, esto significa que se desarrollo una version con las funcionalidades básicas en un corto periodo de tiempo.Después se empezó a añadir a los Baldusers y a pedirles el feedback, y una vez conseguido el feedback se amplio el alcance en base a la información obtenida.

Tanto la calidad del producto final como las ampliaciones realizadas se produjeron mediante las interacciones con los Baldusers. Con cada nuevo feedback recibido se realizaban cambios en Baldugenda y se les pedía opinión.

El aspecto de los Baldusers se ha tratado como un aspecto importante dentro del proyecto para el desarrollo de Baldugenda y su calidad dentro del mercado de las aplicaciones móviles. De esta forma se ha tratado en un tema propio dentro de la memoria mas detalladamente en el capitulo \ref{ch:Usuarios}.

A continuación se hablara de la gestión del alcance dentro del proyecto, mas adelante sobre la dedicación realizada en cada ciclo, después se pasara a la gestión del tiempo empleado y para terminar el capitulo habrá un apartado de conclusiones acerca de la gestión realizada.
\newpage
\section{Gestión del alcance}
\label{secc:Gestión del alcance}

Ya se ha hablado en capítulos anteriores sobre el alcance y cómo ha ido variando dependiendo del ciclo en el que se estuviera. Este aumento del alcance ha sido gracias a los Baldusers que han ido aportando funcionalidades nuevas.
A continuación se enumerará el alcance del primer ciclo:
\begin{itemize}
	\item Creación de asignaturas indicando nombre, posibilidad de añadir enlaces y seleccionando el sistema de evaluación que se seguirá en esa asignatura. 
	\item Creación de exámenes indicando nombre, asignatura, calendario de Google Calendar, fecha y hora.
	\item Manejo básicos de calendarios de la cuenta de Google Calendar del usuario. 
	\item Los usuarios podrán escoger si quieren que se active la notificación en el examen creado dentro de Google Calendar.
	\item Borrado de exámenes desde el examen escogido.
	\item Visualizador de asignatura.
	\item Visualizador de examen.
	\item Lista de asignaturas y buscador dentro de la lista.
	\item Lista de exámenes, agrupados en asignaturas.
	\item Actividad para contactar con el desarrollador mediante email o teléfono.
	\item Uso del API de Google Calendar.
	\item Errores encontrados durante el desarrollo del prototipo resueltos.
\end{itemize}
\newpage
Una vez concluido el primer ciclo se realizó la invitación a los Baldusers y realizaron el feedback. Con el feedback devuelto se aumentó el alcance con las siguientes características:
\begin{itemize}
	\item Modificación del sistema de evaluación, enlaces y nota de una asignatura.
	\item Modificación de fecha,hora,asignatura,calendario y nota de un examen.
	\item Borrado de asignaturas que no tengan exámenes.
	\item Nuevo campo de nota en examen.
	\item Nuevo campo de nota en asignatura.
	\item Cambio de diseño de menú principal.
	\item Errores encontrados por los Baldusers resueltos.
\end{itemize}
Después del segundo ciclo, y después de añadir más Baldusers y recibir el feedback se aumentó de nuevo el alcance:
\begin{itemize}
	\item Nuevo campo de descripción en examen.
	\item Editar examen dentro de la actividad examen y desde la lista de exámenes.
	\item Editar asignatura dentro de la actividad asignatura y desde la lista de asignaturas.
	\item Borrar examen dentro de la actividad examen.
	\item Borrar asignatura dentro de la actividad asignatura y desde la lista de asignaturas.
	\item Exportar Base de datos de Baldugenda a la cuenta de Google Drive del usuario.
	\item Importar Base de datos de Baldugenda desde la cuenta de Google Drive del usuario.
	\item Uso del API de Google Drive.
	\item Errores encontrados por los Baldusers resueltos.
\end{itemize}


\newpage
\section{Gestión de costes}
\label{secc:Gestión de costes}

A continuación se detallaran mediante horas el coste humano que el proyecto ha supuesto.
Se dividirá en gestión del proyecto, aplicación Android, gestión del feedback y memoria.
Se mostrará toda la información de las horas mediante la siguiente tabla:


\begin{table}[h]
\centering
\begin{tabular}{@{}llc@{}}
\toprule
{\bf Categoria} & {\bf Tarea} & \multicolumn{1}{l}{{\bf Dedicación (h)}} \\ \midrule
Gestión del proyecto & \begin{tabular}[c]{@{}l@{}}Reuniones de control\\ Gestión de alcance\\ Adquisición de tecnologías y servicios\end{tabular} & \begin{tabular}[c]{@{}c@{}}9\\ 20\\ 6\end{tabular} \\ \midrule
Aplicación Android & \begin{tabular}[c]{@{}l@{}}Diseño gráfico de interfaz\\ Lógica de la interfaz\\ Aprendizaje API Google Calendar\\ Aprendizaje API Google Drive\\ Google Play\\ Corrección de errores\\ Pruebas\end{tabular} & \begin{tabular}[c]{@{}c@{}}20\\ 90\\ 50\\ 10\\ 4\\ 20\\ 30\end{tabular} \\ \midrule
Gestión de feedback & \begin{tabular}[c]{@{}l@{}}Motivación y promoción\\ Preparación de cuestionarios\\ Entrevistas y síntesis de los comentarios\end{tabular} & \begin{tabular}[c]{@{}c@{}}10\\ 7\\ 12\end{tabular} \\ \midrule
Memoria & \begin{tabular}[c]{@{}l@{}}Redacción\\ Aprendizaje de Latex\end{tabular} & \begin{tabular}[c]{@{}c@{}}55\\ 3\end{tabular} \\ \midrule
 & \multicolumn{1}{r}{{\bf TOTAL}} & {\bf 346}
\end{tabular}
\caption{Tabla de dedicación horaria}
\end{table}
\newpage
\section{Gestión del tiempo}
\label{secc:Gestión del tiempo}

A continuación se detallaran los hitos más importantes dentro del proyecto:

\begin{itemize}
	\item \textbf{22 de enero} Comienzo del primer ciclo
	\item \textbf{24 de enero} Alta de Baldugenda en Play Store
	\item \textbf{23 de febrero} Publicación primer prototipo funcional
	\item \textbf{5 de marzo} Baldugenda integrado con Google Calendar y subida al Play Store
	\item \textbf{26 de marzo al 7 de abril} Invitación a los Baldusers, entrevistas y feedback
	\item \textbf{27 de marzo} Finalización del primer ciclo y comienzo del segundo ciclo
	\item \textbf{23 de abril} Lanzamiento Baldugenda con primeras mejoras de Baldusers al Play Store
	\item \textbf{24 de abril al 8 de mayo} Invitación a los Baldusers tercer ciclo, entrevistas y feedback
	\item \textbf{24 de abril} Finalización del segundo ciclo y comienzo del tercer ciclo
	\item \textbf{21 de mayo} Lanzamiento Baldugenda con segundas mejoras de Baldusers al Play Store, fin de las tareas de implementación y finalización del tercer ciclo.
	\item \textbf{5 de junio} Finalización de la primera versión completa de la memoria del proyecto
\end{itemize}
\newpage
\section{Conclusiones sobre la gestión}
\label{secc:Conclusiones sobre la gestión}

Para finalizar el capítulo sobre la gestión del proyecto realizaremos un breve repaso de los apartados anteriores. Dentro de la gestión del alcance se puede observar que durante la primera fase se tuvo un mayor peso en las tareas de implementación y aprendizaje sobre las APIs de Google, por este motivo es la fase que más tiempo ha llevado dentro del proyecto. En las fases posteriores el tiempo invertido se dividió en la interacción con los Baldusers y las modificaciones que iban pidiendo que se realizaran en la aplicación.

En el apartado de costes se han dividido las categorías de la aplicación de la de los usuarios. Aunque están muy relacionadas, se invirtió un tiempo importante del proyecto en la realización de las pruebas tanto individuales como con usuarios, por este motivo esa tarea es la única en la que también están incluidas las pruebas con usuarios.
Aunque al principio se decidió trabajar con otro procesador de textos, al final acabamos decidiendo trabajar con Latex, ya que todos los aspectos de diseño y estructura serían más sencillos de realizar mediante la plantilla y se llevaría menos tiempo realizarlo.
Un aspecto muy importante para haber llevado correctamente el tiempo de cada una de las tareas, ha sido las anotaciones periódicas de cada una mediante los resúmenes de las reuniones y las subidas al Github, donde se subía una versión estable del proyecto cada 5 horas aproximadamente de trabajo.

El trabajo en fases fue muy útil de la forma en la que está pensado el desarrollo del proyecto. Al trabajar con usuarios, las fases se alargaban más que si se hubiera realizado sin el feedback de estos, aunque el tiempo que se tuvo que esperar a las respuestas de los usuarios se invirtió en probar más a fondo la aplicación, solucionar errores y poder realizar las entrevistas de una forma mas amena y calmada.
El tener separado la planificación de cada fase ha resultado muy útil a la hora de realizar una planificación general del proyecto, teniendo los hitos más importantes marcados, no ha habido ningún problema a la hora de hacer un cronograma.




































% line in order to check if utf-8 is properly configured: áéíóúñ