\section{Conclusiones sobre la gestión}
\label{secc:Conclusiones sobre la gestión}

Aunque ya se ha realizado una conclusión específica en el capítulo sobre la gestión, me gustaría hablar sobre este tema. La gestión usada durante el proyecto vino marcada por el uso de fases, periodos en los que se iban realizando implementaciones y se añadían usuarios para que realizaran pruebas. Al ser fases cortas, el seguimiento que se tenía del alcance era mucho más sencillo, ya que se podía comprobar si se habían realizado las funcionalidades para esa fase. En cuanto a las horas dedicadas, al poder dividir el proyecto en pequeños periodos, se podía comprobar el tiempo que había llevado la realización de esas tareas, e ir llevando la cuenta del tiempo que se tenía invertido hasta ese momento. 

El trabajar con usuarios también fue algo que produjo costes que no se tenían pensados en un principio, tales como realizar reuniones con los usuarios o motivarles para que realizaran el feedback antes de una fecha en concreto, esto fue algo que había que realizar continuamente. Este motivo fue uno por los que se decidió no presionar tanto a los usuarios a que realizaran las actas y las encuestas, y se enfocó más a escucharles lo que querían decir sin necesidad de forzarles. Ellos eran los que me estaban haciendo el favor a mí, y no tenía que ser una obligación para ellos realizar las pruebas. Hubo momentos en los que las pruebas se alargaron por las fechas, ya que los usuarios no respondían el feedback necesario, así que se decidió ir realizando modificaciones pedidas en fases anteriores mientras los usuarios respondían.

El tema de los usuarios es algo que me ha marcado mucho, y es algo que antes del proyecto se me había olvidado y que había que darle importancia. Un ingeniero es aquél que implementa soluciones a problemas que afectan a la vida cotidiana de la sociedad. En la definición de ingeniero un aspecto que no hay que olvidar es el apartado de la sociedad, por muchas funcionalidades que se implementen en una aplicación, si en ese momento los usuarios no las ven necesarias, esa aplicación no tendrá el futuro que se había pensado desde un principio.


