\section{Experiencia personal}
\label{secc:Experiencia personal}

Una vez concluida la parte de implementación del proyecto me pongo a pensar sobre todas las cosas que me han resultado laboriosas a la hora de implementar la aplicación pero después pienso que si no llega a ser porque me costó realizarlas ahora mismo no estarían plasmadas en esta memoria ni tampoco estarían en mi cabeza para siempre. Cuando las cosas se superan fácil no se les da importancia y se sigue adelante, en cambio cuando como es el caso de un desarrollador de software, cuando se consigue implementar algo que ha llevado horas de dedicación y ha costado un esfuerzo que no se puede medir en horas ni en dinero, la sensación que se tiene es de felicidad. Muchas cosas de este proyecto no se me olvidaran ya sea porque me resultaron interesantes a la hora de usarlas o como se ha comentado antes por la dificultad que supusieron. Tanto el uso de los ciclos como trabajar con usuarios me hicieron darme cuenta que una aplicación no es solo un software que se crea con una finalidad y ya está, sino que hay personas detrás que tienen que conseguir que ese producto sea atractivo para los usuarios. 

El uso de las APIs también es uno de los temas que seguro no se me olvidara jamás, al principio en la idea del producto que quería diseñar no estaban presentes las APIs pero después de ver la app desde otro punto de vista me di cuenta que sí serían necesarias, ya que sin esos servicios la aplicación no tendría futuro.

El tema de los usuarios es algo que me ha marcado mucho y es algo que antes del proyecto se me había olvidado, y que había que darle importancia. Un ingeniero es aquel que implementa soluciones a problemas que afectan a la vida cotidiana de la sociedad. En la definición de ingeniero un aspecto que no hay que olvidar es el apartado de la sociedad, por muchas funcionalidades que se implementen en una aplicación, si en ese momento los usuarios no las ven necesarias, esa aplicación no tendrá el futuro que se había pensado desde un principio.
