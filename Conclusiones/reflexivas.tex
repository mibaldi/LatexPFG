\section{Experiencia personal}
\label{secc:Experiencia personal}

Una vez finalizada la parte de desarrollo del proyecto, me pongo a pensar sobre todas las tareas que me han resultado laboriosas a la hora de implementar la aplicación, pero después pienso que si no llega a ser porque me costó realizarlas, ahora mismo no estarían plasmadas en esta memoria ni tampoco estarían en mi cabeza para siempre. Cuando las dificultades se superan fácil no se les da importancia y se sigue adelante, en cambio en el caso de un desarrollador de software, cuando se consigue implementar algo que ha llevado horas de dedicación y ha costado un esfuerzo que no se puede medir en horas ni en dinero, la sensación que se tiene es de felicidad. Muchos aspectos de este proyecto no se me olvidarán, ya sea porque me resultaron interesantes a la hora de usarlos, o como se ha comentado antes por la dificultad que supusieron. Tanto el uso de las fases como trabajar con usuarios, me hicieron darme cuenta que una aplicación no es sólo un software que se crea con una finalidad y ya está, sino que hay personas detrás que tienen que conseguir que ese producto sea atractivo para los usuarios. 

Todos estos aspectos junto con lecciones aprendidas durante la carrera han sido de mucha utilidad. El trabajo en grupo y los proyectos que se tuvieron que realizar en las distintas asignaturas, han servido para que en un futuro cuando este trabajando pueda recordar todos estos aspectos y darme cuenta que durante el tiempo que he pasado en la universidad me ha hecho crecer como persona.
Cuando se esta realizando las asignaturas en primero y segundo de carrera, uno piensa que no le aportan nada o que sería mas útil si se diera otro tipo de materia. Pero una vez finalizada la carrera se ven de distinta forma todas esas asignaturas que servían para formar alumnos que no tenían ninguna experiencia en esos lenguajes. Gracias a esas enseñanzas ahora el aprendizaje de los lenguajes de programación o el uso de nuevas tecnologías me resulta mas sencillo que si no me hubieran formado de esta forma.

Gracias a todo lo anterior me doy cuenta que se ha conseguido lo que se buscaba desde un principio, moldear a los alumnos que entraban a la carrera interesados en informática, y hacer de ellos unos profesionales en el área de la ingeniería informática.



