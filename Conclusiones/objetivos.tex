\section{Conclusiones sobre los objetivos y tecnologías desplegadas}
\label{secc:Conclusiones sobre los objetivos}

Para empezar en esta sección, hablaremos del modelo seguido. Como ya se ha explicado en distintos capítulos el modelo de desarrollo seguido ha sido el de desarrollo en espiral. En mi opinión, este tipo de modelos está bien pensado para proyectos como éste, en el que el desarrollador no conoce las tecnologías implicadas,ni las necesidades a las que dará cobertura, y no puede saber cuánto tiempo le va a llevar realizar cada implementación. Al dividir el proyecto en pequeñas fases, el control del tiempo invertido en cada implementación es más fácil de llevar, y también es útil para centrarse en ciertas partes del programa en cada fase. En el caso de Baldugenda, al dividir en tres fases se consiguió que los riesgos que no se conocían como es el caso del API de Google Calendar y su implementación dentro de las aplicaciones Android se viera reducido, ya que al centrarse durante una fase entero a que funcionara correctamente, después en las siguientes fases no se tuvo que estar volviendo a pelear con el API. Es una buena forma de desarrollar si no se conocen las mencionadas tecnologías, en otras situaciones, como es el caso en el que se conocen las tecnologías que se van a usar, y se haya realizado un proyecto parecido, el estar trabajando con pequeños prototipos y realizando modificaciones tan pequeñas puede llegar a retrasar el producto.

Por otra parte, están las tecnologías usadas: Por un lado en Baldugenda se usó Android como sistema operativo de dispositivo móvil; por otro lado para el desarrollo de la aplicación se decidió usar Android Studio; asimismo para el control de versiones se usó Github; y la memoria se realizó en Latex. De las tecnologías usadas, todas se habían usado con anterioridad menos Latex. Después de finalizar la implementación de Baldugenda,  la sensación que me dejan estas tecnologías es muy buena, ya que se nota que son tecnologías nuevas como es el caso de la primera versión de Android Studio, que van por buen camino. 

El uso de Github también ha sido un factor importante, al principio del proyecto no se usaba mucho esta plataforma, ya que se trabajaba de manera individual. En cambio, al comenzar el proyecto se decidió darle más importancia y usarla tanto para la parte de código como para la de documentación. Al usar un control de versiones, podía estar tranquilo de que lo realizado hasta ese momento no se perdería, de todas formas se realizaron copias en local y en la nube de Google Drive.
\newpage
En el apartado de la aplicación, el desarrollo que se ha seguido a la hora de implementarla ha sido bueno, al estar subiendo periódicamente versiones estables se conseguía tener la certeza que lo realizado hasta el momento funcionaba, y que aunque se pudieran mejorar las implementaciones, se tenía un \acrshort{mvp} con el que seguir trabajando.

El uso de las APIs es uno de los temas que seguro no se me olvidara jamás. Al principio en la idea del producto que quería diseñar no estaban presentes las APIs, pero después de ver la app desde otro punto de vista, me di cuenta que sí serían necesarias, ya que sin esos servicios la aplicación no tendría futuro.
De las funcionalidades implementadas, las que más costaron fueron las que tienen que ver con el Google Calendar, el problema de trabajar con un API que no se conocía, fue que al querer implementar el API, primero se optó por REST, consiguiendo de esta forma alargar la fase sin conseguir resultados, después se decidió hacer uso de las librerías del API y una vez usadas las librerías ya todo fue más sencillo. A parte del API de Google Calendar, surgieron unos problemas al trabajar con distintas versiones de bases de datos, que una vez resuelto el primer conflicto encontrado, ya fueron únicamente modificaciones para las siguientes versiones.
Dentro del problemas de las APIs de Google, llegó el momento de Google Drive. A la hora de realizar la implementación no hubo tantos problemas como con Google Calendar, pero esta vez el problema fue con el ciclo de vida de las actividades en Android y sus hilos de ejecución. Se consiguió resolver en menos tiempo de lo esperado.
