\chapter{Desarrollo de apps tipo Baldugenda}
\label{ch:desarrollo}
Para empezar, hay que explicar a que nos referimos cuando estamos hablando de una app tipo o de la categoría de Baldugenda. Al registrar Baldugenda en el Play Store de Google hay que definirla en una categoría, qué en este caso es la de educación. Aparte Baldugenda es un tipo de aplicación que maneja los calendarios del usuario y le permite crear eventos en éstos. Un ejemplo de aplicaciones tipo Baldugenda para Android podrían ser:  \href{https://play.google.com/store/apps/details?id=com.clawdyvan.agendadigitalaluno&hl=es}{Agenda del estudiante} y \href{https://play.google.com/store/apps/details?id=ugr.cursoandroid.agendauniversitaria&hl=es}{Agenda Universitaria}.

El capítulo de desarrollo de apps tipo Baldugenda se ha dividido en dos partes: una referente a Google y la importancia que tiene en las aplicaciones Android, y otra parte, donde se hablará más sobre los aspectos importantes referente al mundo Android dentro del tipo de aplicación que se está desarrollando, como podría ser el uso de calendarios o las notificaciones.

% El siguiente es el texto que aparece en el encabezamiento del proyecto fin de grado (si se quiere que sea distinto al nombre del capítulo))
%\chaptermark{PFG}


\section{Google}
\label{secc:google}

Un punto importante en lo que al desarrollo de aplicaciones móviles se refiere, es el uso que los usuarios dan a las aplicaciones, hay muchas aplicaciones en el mercado de la tecnología móvil, pero las aplicaciones punteras que salen adelante son las que tiene compatibilidad con otras aplicaciones que usan habitualmente los usuarios, como puede ser el correo, las redes sociales, o sin necesidad de ese tipo de funcionalidades. Estos últimos las usan porque son realmente originales en su ámbito.
Google ha hecho un negocio de las aplicaciones que facilita a los desarrolladores a que  realicen aplicaciones proporcionando APIs de sus productos, y permitiendo que suban las aplicaciones a su tienda.
En este apartado se detallará el uso de Google dentro de aplicaciones del tipo de aplicación que es Baldugenda, una aplicación que controla calendarios, realiza backups de información  y tiene control de cuentas de Google.

\subsection{Google Play}
\label{subsecc:Google Play}

Para empezar Google proporciona la posibilidad de compartir  las aplicaciones por medio de Google Play.

Para poder desarrollar aplicaciones, y colgarlas en el Play Store, hay que realizar un pago de 25\$, con ese pago se consigue que la cuenta de Gmail escogida pase a ser de desarrollador, lo que permite crear aplicaciones y subirlas, también se puede monetizar esas aplicaciones por medio de publicidad.

Una vez se ha realizado el pago la creación de una aplicación se separa en dos partes principales.

\begin{itemize}
\item Desarrollo mediante Eclipse o \gls{Android Studio}
\item Subida al Google Play
\end{itemize}

Para programar en Android la mejor opción es la de Android Studio, ya que genera los ficheros y el manifest del proyecto sin tener que estar modificándolo todo el rato.
Si se escoge la opción de Eclipse, Google ya no da el soporte necesario y es más complicado programar en él.
Dentro de Android Studio hay varias posibilidades a la hora de crear un proyecto, y depende de cuál se escoja funcionará para distintos dispositivos.
Cuando se ha creado el proyecto, y se ha conseguido probar sin errores, hay que firmar el apk, para esto Android Studio permite generar un apk firmada de una manera fácil.

\newpage
\begin{figure}[H] 
  \begin{center} 
    \scalebox{1}{\includegraphics{figs/CreacionKey.png}} 
    \caption{Ventana de creación de Key} 
    \label{fig:CreaciónKey} 
  \end{center} 
\end{figure}

Con la ventana anterior se genera un fichero que se usará como llave para firmar la aplicación. En el caso de necesitar usar APIs de Google se necesitará la contraseña cifrada en Sha1 para enlazar el API con la aplicación.


La parte más pesada después de desarrollar la aplicación es el papeleo con Google, para subir cualquier aplicación hay que rellenar muchos campos dentro del formulario para que Google permita subir la aplicación. En estos campos la mayoría tienen información de la aplicación como imágenes del icono que usaremos, el tipo de aplicación que es, qué tipo de gente puede usarla, y otras preguntas parecidas a las anteriores.

Una vez Google permita colgar la apk, dará tres opciones,  en producción, beta  o alfa. Estas opciones limitaran la cantidad de gente que se puede descargar la aplicación. Ademas ayuda a tener una versión de la aplicación funcionando mientras se está probando una versión superior en otro tipo de subida.
En el caso que se quiera compartir en alpha o en beta, hay dos opciones para agregar a mucha gente a la vez, la primera opción es la que se usó durante la primera fase de las pruebas con los Baldusers, que es creando una comunidad de Google plus, y dentro poniéndoles el link para que se la descargasen. O la segunda opción que se usó en la segunda fase de las pruebas, que se creó un grupo de Google groups, y se añadió los e-mails de la gente, dentro de la invitación se les añadió el link de descarga. Aquí se limitó los permisos de la gente del grupo para que no pudieran crear nada ni responder a nadie.

La primera opción dio muchos fallos, ya que había gente que por el tipo de cuenta que usaban en su universidad o instituto no se les permitía usar Google plus así, que no podían asociarse al grupo de usuarios por esa forma, aparte si no se tenía cuenta en Google plus, esta opción les obligaba a crearse una cuenta con el coste de tener que explicarles paso a paso cómo debían hacerlo si no tenían muchos conocimientos de las redes sociales.
La segunda opción, y en mi opinión la más útil, fue la de crear los grupos mediante Google Groups. El Balduser sólo recibía un email con 2 links, uno para aceptar la invitación y otro para unirse a las pruebas de la descarga de la aplicación. No tenían que hacer nada más. Lo bueno de esta opción es que se estaba trabajando con el grupo de Baldusers como si fuera una lista de e-mails, y en el caso de tener que mandar una información general solo había que redactar un correo de grupo.

Una vez seleccionado los grupos que tenían acceso a la aplicación en fase alpha, el siguiente paso era subir la apk.
La subida a Google play no es inmediata, por esta razón, cuando se subía Baldugenda solían pasar un par de horas hasta que la descarga estuviera disponible.
Unos puntos importantes a la hora de generar las aplicaciones son: el peso de la aplicación, los dispositivos a los que va dirigido y los permisos que se pedirán a los usuarios.
Los dos primeros puntos están vinculados, ya que si se escoge un grupo muy amplio de dispositivos, tanto por pantalla como por versión de Android, se estará ante una situación en la que para que se vea correctamente en todos los dispositivos, habrá que introducir iconos de diferentes tamaños y generar vistas dependiendo del dispositivo.

Siempre hay que pensar que no es lo mismo desarrollar una aplicación web que una aplicación móvil. Los dispositivos móviles no disponen de mucha capacidad de almacenamiento, así que cuanto menos pese la aplicación, más posibilidades hay que no dé problemas a la hora de instalarla.
Aparte de los iconos y las vistas, también dependemos de las librerías, cuanto más alta es la versión de Android más peso añade al programa a la hora de la instalación, porque le añade funciones nueva de las librerías aunque no las hayamos usado, como por ejemplo un cambio de color al pulsar un botón, que aunque en nuestro código no se pueda ver, ni en las versiones anteriores tampoco, cuanto mayor es la API más funciones agregaría.
Google limita el peso de las apk que se pueden subir a Google Play en 50 mb, a partir de ese peso se pueden subir extensiones para la aplicación.

\subsection{Uso de APIs}
\label{subsecc:Uso de APIs}

Ya se ha explicado cómo funciona la subida de una aplicación a Google play, pero eso no es lo único que Google proporciona de ayuda a la hora de desarrollar la aplicación y distribuirla.
Google proporciona sus APIs propias para que se vinculen los servicios que ofrece Google con la aplicación que se quiere desarrollar. Si la aplicación que se quiere desarrollar no va a realizar un gran uso de los servicios de Google no hay problema, el problema surge cuando se alcanza el límite de usos de la API, y Google exige pagos para seguir usándolo al mismo rendimiento.
Aunque a priori las cuotas parecen altas, siendo como es el caso de Google Calendar API \cite{GCalendar} de 1.000.000 de solicitudes por día, si una aplicación se quisiera comercializar,  habría que estar al tanto de cuántas solicitudes se realizan, y si se necesita aumentar la cuota, puesto que al tener pública la aplicación, cualquiera en el mundo podría descargársela.

Para el uso de estas APIs, Google ofrece distintas posibilidades dependiendo de la API, ya que hay algunas que necesitan autenticación para funcionar, como puede ser el caso de Google Calendar frente a otras como Google maps, que lo único que necesitan es la credencial de tu aplicación para llevar al día la cuota de la API.
Este apartado de credenciales de las APIs es un tema bastante difícil a la hora de usar los servicios de Google, y donde más problemas se suelen tener. Google lo explica en su web paso a paso cómo configurar las credenciales. Aunque varias veces habrá que generar credenciales que no se sabe para qué sirven porque las que dice Google que se generen no dan acceso a la API.

Dentro de la consola de desarrollador de Google, es donde hay que añadir los certificados del API, hay distintas pestañas, algunas dan información del uso que se ha dado del servicio, y otras dan la posibilidad de agregar más servicios al proyecto.
Se pueden encontrar servicios tales como el Google Calendar o el Drive que se han usado para la parte de la administración de calendarios y el backup dentro de Baldugenda, como también podemos encontrar servicios de youtube, traductor, publicidad, mapas. 

Todos los productos que Google quiere vender están al alcance de los desarrolladores.
Lo bueno que tiene usar los servicios de Google es que todo aquél que se descargue la aplicación, habrá usado alguna vez esos servicios, o tendrá la aplicación instalada en su dispositivo aunque no la use. Si se usaran servicios de terceros que no fuera Google, se tendría que pedir al usuario que se creara otra cuenta o que se instalara otra aplicación para poder sacar todo el partido a la aplicación que se está creando.

La finalidad de usar servicios de terceros de empresas importantes en los dispositivos móviles es que los usuarios usen el producto que se ha desarrollado, y que no se lo desinstalen, uno de los motivos por los que la gente desinstala aplicaciones pequeñas es que les exigen estar duplicando información que quieren tener centralizada. Un ejemplo son los eventos del calendario. Hay muchas aplicaciones que usan el calendario del móvil pero que no dan soporte a calendarios online como el de Google Calendar, en este caso, si el usuario quiere crear un evento con esa aplicación, no puede modificarlo desde el ordenador y le limita a usar siempre el móvil con ese evento.
Pasa lo mismo con las notas o con los ficheros al querer guardarlos en los móviles, generando así una necesidad para el usuario de tener un espacio de almacenamiento enorme si quiere sacar fotos y descargase cualquier información al móvil.

\subsection{Uso de calendarios}
\label{subsecc:Uso de calendarios}

En el apartado anterior se ha hablado de las APIs,  una de las que proporciona Google es la de Google Calendar. En las aplicaciones parecidas a Baldugenda el uso de Google Calendar no es habitual, la mayoría generan los horarios en calendarios propios de la aplicación, y te muestran fechas del calendario sólo accesibles desde el móvil.
Google al proporcionar el API de Google Calendar, y al estar tan vinculado a Android para descargar las aplicaciones, proporciona una relación con el calendario del móvil casi perfecta tanto para visualizar los eventos como con las notificaciones de dichos eventos.
Con el servicio del calendario, Google permite realizar todo lo que se podría hacer desde el servicio web de Google Calendar pero desde la aplicación que creemos.

A la hora de desarrollar una aplicación usando este API, Google propone dos formas para empezar a desarrollar: una usando el API \acrshort{rest}, y la otra usando la librería Java que nos proporcionan con los métodos.
El problema en esta parte es que si se tiene algún problema con el API REST, el código que dan de ejemplo es demasiado pequeño si se quiere realizar funcionalidades complejas. Y en el código de ejemplo que Google tiene en \gls{Github}, sólo se encuentra el API usando la librería.
Tanto si se usa la librería o si se decide uno por usar REST, Google ha separado las funciones en apartados para que sea lo más cómodo posible encontrar lo que se necesita.

Algo importante que hay que saber a la hora de programar con este API, es que se necesita credenciales OAuth2 y credenciales públicas para que den permiso a la aplicación que se desarrolle a que use el API de Google Calendar.
En el apartado anterior se ha explicado cómo añadir credenciales al proyecto, una vez añadidas, lo que queda es ver qué funciones se necesita usar para crear el evento o para visualizar los calendarios, y ya se podrá realizar modificaciones en los calendarios de Google.

Cada API es un mundo, y este API no se queda atrás, las principales funciones que todo usuario habitual usa, como crear un evento, visualizar los calendarios, etc. son sencillas de usar, el problema surge cuando se empieza a complicar el caso de uso, y el uso del API resulta engorroso.
En el caso de Baldugenda se mantuvo el apartado para la gestión de calendarios que proporcionaba Google, ya que servía al usuario para crear calendarios específicos para los eventos de la aplicación. Pero, dio problemas al no sincronizarse automáticamente al calendario del dispositivo, siendo ese el motivo por el cual muchos usuarios pensaron que no se creaba bien el evento, y cuando se ponían en contacto conmigo, se les explicaba cómo actualizar el calendario, les aparecían todos los eventos e incluso duplicados porque habían generado 2 o 3 exámenes con distinto nombre pero haciendo referencia al mismo examen.

\subsection{Debugging}
\label{subsecc:Debugging}

El apartado de debugging es un punto importante en todo proceso de desarrollo de aplicaciones, tanto de móviles como de cualquier otro tipo, pero en el caso de los móviles se vuelve un trabajo muy difícil llevarla a cabo cuando ya está en prueba por los usuarios, ya que los errores que se produzcan no los podrá ver el desarrollador, a menos que la aplicación tenga implementado funciones para esos casos.
Cuando se está desarrollando la aplicación, Google proporciona mediante Android Studio una excelente ventana de debugging para seguir paso a paso, y  línea por línea el recorrido que realiza la aplicación.
Se puede añadir líneas de log especificando si es de error, de peligro o directamente de información. 

Aparte de estos logs, se puede seguir el valor que tiene una variable en todo momento desde que nace hasta que muere con la actividad.
El problema de esta forma de debugging, es que cuando la aplicación deja de estar conectada al ordenador y al Android Studio, ya no podemos seguir la pista de lo que pasa.
Google proporciona información de los usuarios que se han instalado la aplicación en la ventana donde se ha tenido que publicar dicha aplicación. Aunque esta información es mínima, incluye información útil como las versiones de los dispositivos que se la han instalado, la operadora del dispositivo, el país desde donde se la han descargado y el lenguaje de la instalación del dispositivo.
\newpage
\begin{figure}[H] 
  \begin{center} 
    \scalebox{1}{\includegraphics{figs/UsoPorPaises.png}} 
    \caption{Uso por Países de Baldugenda Console Developer} 
    \label{fig:UsoPorPaises} 
  \end{center} 
\end{figure}

Como se puede observar en la figura, las instalaciones actuales son 13, y una desde Alemania, que es el Balduser que está de erasmus.

\begin{figure}[H] 
  \begin{center} 
    \scalebox{0.7}{\includegraphics{figs/InstalacionesActuales.png}} 
    \caption{Instalaciones actuales de Baldugenda Console Developer} 
    \label{fig:InstalacionesActuales} 
  \end{center} 
\end{figure}

Las opciones de gráficos facilitan saber cuántos usuarios han actualizado la versión nueva, y cuántos se mantienen en versiones anteriores.
También posibilita verlo mediante una escala de tiempo


\begin{figure}[H] 
  \begin{center} 
    \scalebox{1}{\includegraphics{figs/EscalaTiempo.png}} 
    \caption{Escala de tiempo Console Developer} 
    \label{fig:EscalaTiempo} 
  \end{center} 
\end{figure}

Aparte de todo esto, Google promociona el uso de Google analytics, un servicio web para recopilar datos de interacción del usuario en las aplicaciones que se generen.
En la aplicación de Baldugenda, se decidió usar una API de terceros llamada Splunk Mint, que se detallará más adelante para la labor de recopilar información sobre los fallos en la aplicación.
Al realizar acciones de debugging hay que ponerse en todas las situaciones posibles, en las que se pueda llegar a estar en la aplicación. Al ser una aplicación móvil no siempre se tendrá la misma cobertura de Internet o las cuentas de Google podrán modificarse.

También puede darse el caso que el móvil falle, y se tenga que borrar,  todas estas situaciones hay que tenerlas en cuenta. Una de las situaciones que no se suele tener presente cuando no se está acostumbrado a realizar aplicaciones para móviles, es que hay dos posiciones, y que dependiendo de qué posición tenga el móvil, puede que la información se vea distinta o directamente no se vea.
Puede darse el caso que falle la aplicación en algún punto crítico, como puede ser la realización de una escritura en la base de datos o en una llamada al servidor, esas situaciones suelen ser muy comunes al no estar conectado a Internet  todo el rato.

Como al principio no se puede pretender que se sepa en dónde va a fallar en cada momento; viene bien instalarse la opción que facilita Google con su Google analytics u otras opciones de terceros como Splunk mint.
Ya que mientras los usuarios que están haciendo pruebas les falla, se puede configurar para que te llegue al correo, el motivo por el que la aplicación ha dejado de funcionar y en qué líneas se ha producido el fallo, gracias a esta información ya se pueden sacar conclusiones de por qué ha fallado y cómo se podría solucionar. Vale de poco pasarse horas y horas probando casos que igual se producen 1 de cada 100.000 veces,cuando no hemos tenido en cuenta algún caso que una tercera parte de los usuarios van a verse afectados.

Los usuarios, la mayoría no sabrán decir lo que les ha pasado cuando ha fallado la aplicación, ni tan siquiera lo que estaban haciendo cuando ha dejado de funcionar, por este motivo tener a un dispositivo que nos diga al menos que botón han pulsado es útil.
 


\subsection{Notificaciones}
\label{subsecc:Notificaciones}

Google proporciona un servicio de notificaciones push muy potente llamado Google Cloud Messaging (GCM), este servicio permite hacer que aparezcan esos mensajes en la pantalla de arriba del móvil cuando llega un mensaje, o cuando nos piden vidas en los juegos de las redes sociales.
Gracias a estas notificaciones, no se necesita tener la aplicación conectada, ya que trabajan sobre un servidor y es éste quien se pone en contacto con nosotros.

En el caso de Baldugenda no se han usado este tipo de notificaciones, el uso de notificaciones se ha limitado a las que vienen con los eventos que genera Google Calendar.
Al crear un evento, si el calendario tiene la opción de realizar notificaciones, cuando se vaya a producir el evento, éste se creará sin ninguna alarma programada, pero el propio calendario avisará al usuario con una notificación al móvil, o a la versión web si tiene la versión web de Google Calendar abierta.
Para que se produzca la notificación en el móvil, el calendario tiene que estar sincronizado con éste y actualizado para que se tenga cargado los eventos nuevos desde la última sincronización, si no se da esa situación, la notificación no saltará.
Se decidió no meter alarmas en los exámenes porque eso obligaría al usuario a tener que estar metiéndose en la aplicación cada vez que quisiera desactivar la alarma, en cambio al realizarlo mediante Google Calendar y la creación de eventos, el usuario es libre de realizar las modificaciones en Google Calendar, y tener los exámenes en un calendario con las notificaciones activadas.

\subsection{Backup}
\label{subsecc:Backup}

La necesidad del Backup es algo importante en cualquier aplicación en la que el usuario tiene que meter algún dato, a nadie le gusta perder información que ha estado invirtiendo su tiempo escribiendo. Los móviles suelen romperse o estropearse y en esas situaciones, perder todo supondría un grave problema, para esas situaciones una copia de seguridad es el mejor salvavidas.
Para el Backup, Google tiene opciones tales como las propias implementadas en Android: el backup API, que, aunque hemos separado las explicaciones en Google y Android, Google compró Android en 2005 y muchas funcionalidades que se pueden encontrar en Android están vinculadas con Google.

También tiene otras opciones más simples, como puede ser usar una opción de respaldo en la nube de Google Drive\cite{GDrive}, transparente para el usuario.
En la primera opción, la ventaja que tiene es que el usuario no tiene que realizar ningún paso para realizar la copia de seguridad, es el propio desarrollador quien prepara todo, pone la clave de la aplicación que se usa en todas los servicios Google para vincular la aplicación con el servicio, y después el sistema realiza el backup donde el desarrollador lo escoja. Luego Google guarda la información en la nube sin que el usuario pueda acceder a ella, solo se podrá acceder a esa información por medio de la aplicación que ha realizado el backup.

\begin{figure}[H] 
  \begin{center} 
    \scalebox{1}{\includegraphics{figs/Drive.png}} 
    \caption{Google Drive Google} 
    \label{fig:Drive} 
  \end{center} 
\end{figure}

La segunda opción parece que tiene más problemas que ventajas, pero dependiendo de cómo se use puede ser incluso más útil que la primera.
En esta segunda opción el usuario puede ver en todo momento dónde está su información alojada, y si no se fía de tenerla en Google Drive, puede descargarla y guardarla en su ordenador o mandarla por e-mail.
El problema de tener la copia de seguridad tan accesible es que el usuario puede modificar el fichero, esto podría ser un problema pero con una buena seguridad a la hora de importar la base de datos, esas situaciones no tendrían que repercutir en la aplicación.
Al estar implementando la parte de backup en Drive, se encontró que por cada aplicación que se instala, los desarrolladores pueden usar una parte del Google Drive si se les otorga permiso que ni los propios usuarios podrán tocar, y sirve como carpeta de preferencias, la cual al borrar la aplicación desaparecería.

En Baldugenda esa carpeta no se usó, ya que ya se había empezado a usar el sharedpreferences de Android, pero si en alguna aplicación futura se quisiera reducir el espacio que ocupa la aplicación, esta carpeta podría resultar de utilidad.
Otra de las ventajas de realizar la copia en Google Drive es que si en un futuro se quisiera compartir información entre usuarios de Baldugenda, sólo se tendría que modificar la opción de importar y agregar una opción de juntar agenda existente. Con esto se podrían compartir tanto exámenes como asignaturas y el usuario no tendría que estar metiendo las.

El uso de Google Drive es sencillo ya que tiene tres formas de usarlo y ofrece documentación para cada una de las formas: mediante REST, mediante la librería Java o mediante el sdk de Drive.
El ejemplo que ofrece Google es con el sdk de Drive, Baldugenda tiene implementada esta opción en el código de Github.
Google Drive ofrece muchas posibilidades a la hora de crear las carpetas y subir los ficheros, se pueden crear ficheros usando tus propias actividades en Android y rellenando todom o puedes usar una actividad que proporciona Google Drive para crear y seleccionar ficheros.

Para el usuario es mucho más cómodo mostrarle algo que le resulte familiar a la hora de hacer elecciones, y por eso se optó por usar las actividades que proporciona Google Drive, aparte de que tiene un diseño refinado y una cantidad de opciones y posibilidades asombrosas.
Se dieron ciertos problemas a la hora de realizar el backup con Google, ya que al crear carpetas en Drive o ficheros no se producen inmediatamente, y hay situaciones en los que por diversos motivos el servidor está ocupado y tarda un poco más, en esas situaciones el identificador de la carpeta no era accesible desde el momento que se generaba y tardaba, esto hacía que fallara al momento de seleccionar la carpeta donde el sistema subiría el fichero.
Por ese motivo se decidió separar la parte de la selección de la carpeta de la parte de exportación de la BD.

\subsection{Problemas al desarrollar con Google}
\label{subsecc:Problemas al desarrollar con Google}

La mayoría de los problemas que se encuentran cuando se está trabajando con Google se debe al uso de los servicios de Google dentro de la aplicación.
Cuando se intenta implementar algún servicio de Google, hay que leerse muy bien el manual del servicio y comprobar que esté actualizado. Google tiene un poco desordenado su documentación de los servicios, la ha separado tanto y con tantos enlaces que a veces llega a ser confuso.
Al principio Google explica dentro de su web de APIs qué utilidades tienen y como sacarles partido. Para ello separa cada API y hace una especie de guía, y después pone un apartado para que se empiece a programar, y aquí es donde empieza la confusión, Google suele añadir enlaces a muchos sitios dentro de su documentación para explicar lo más básico del manual, por este motivo hay que tener muy claro donde se tiene que buscar la información.

Una manera de hacer frente a este problema es buscar tutoriales por Internet o video tutoriales donde implementen ese API, y de ahí seguir con las funciones propias que tendrá tu aplicación.
El apartado de los credenciales es igual, el punto que mayores problemas causa a la hora de hacer que funcione una API al principio, dependiendo qué modo se use, o si se tiene que activar unas credenciales u otras. Aparte cada API necesita un tipo de credencial distinta dependiendo a qué recursos quiera acceder la aplicación.
Por ese motivo, lo mejor es familiarizarse con la API fuera del proyecto, creando un proyecto pequeño y haciendo que funcione la API por separado y después incluirlo al proyecto que se quiere implementar.
Google dispone de ejemplos en Github de sus APIs, el problema es que esos ejemplos no suelen estar actualizados o no están en todos los modos en los que Google lo ofrece, por ejemplo el API de Google Calendar que se usó en Baldugenda. Google ofrece un acceso a esa API por medio de REST, al ir a buscar ejemplos para esa API se dieron casos de todo tipo, al ser un API que ya tiene distintas versiones, había ejemplos de cada una de ellas, pero de API REST para Android pocos o que no explicaban bien el código. Y en el Github de Google el único ejemplo que había para el API de Google Calendar era haciendo uso de la librería en Java.

Aparte del uso de las APIs en Google, otros problemas que han surgido han sido al compartir el enlace de descarga de la aplicación.
La primera opción para compartirlo fue por medio de una comunidad de Google plus,  el fallo que se encontró fue que había algunos Baldusers que no tenían cuenta en Google plus y no querían creársela, así que no podían acceder a la descarga de la aplicación.
Para esos casos se usaron los grupos de Google para invitarlos y que pudieran descargarse la aplicación desde ahí.
Algunos Baldusers experimentaron problemas al actualizar Baldugenda, ya que hay un error en Google Play reportado por la comunidad de desarrolladores de Google, que sucede al corromperse la información de las cuentas vinculadas al dispositivo móvil, por ese motivo Google no permite descargar aplicaciones y exige mucho más espacio del requerido.

Para solucionar el problema se estuvo investigando, y la solución era quitar la cuenta de Google del dispositivo y volverla a poner desde cero.
Con eso el Google play cargaría de nuevo la lista de aplicaciones y sus actualizaciones y funcionaría la descarga.
El problema ocasionaba que se les exigía un espacio de memoria altísimo para 2 Mbs que pesaba la aplicación, y al no tener espacio suficiente les daba error al actualizar.


\begin{figure}[H] 
  \begin{center} 
    \scalebox{1}{\includegraphics{figs/ErrorGoogle.png}} 
    \caption{Error Google Play} 
    \label{fig:ErrorGoogle} 
  \end{center} 
\end{figure}

Ahora se hablará de problemas más específicos con respecto al calendario y al Drive.

\subsubsection{Problemas Google Calendar}
\label{subsubsecc:Problemas Google Calendar}

Un problema que no se pudo solucionar mediante código fue el problema de la sincronización de Google Calendar con el dispositivo móvil. La idea principal era que cuando se realizaban nuevos eventos en el calendario, estos eventos se sincronizaran automáticamente con el calendario del dispositivo, si el Balduser no tenía activada la sincronización por tiempo, y la tenía manual, no le aparecería en su móvil a menos que le diera a sincronizar.
Para solucionarlo, se les comunicó como hacerlo a los Baldusers y los que lo vieron necesario lo usaron.
Otro problema que se encontró al implementar las opciones de calendario fue el uso de la autenticación OAuth2 \cite{Oauth} que requiere el API.
Las explicaciones eran muy confusas a la hora de configurar, ya que las ventanas y las opciones que aparecían en la guía de Google eran de versiones anteriores a las configuraciones que tenía que realizar yo.
La forma de solucionarlo fue ver vídeos donde otros desarrolladores creaban credenciales para APIs de Google que requirieran ese tipo de credenciales, y haciendo pruebas se consiguió que empezara a subir la cuota de uso.


\subsubsection{Problemas Google Drive}
\label{subsubsecc:Problemas Google Drive}

Google Drive no dio tantos problemas como en su momento Google Calendar, aún y todo se tuvieron algunos inconvenientes a la hora de implementar la API.
Para facilitar el trabajo con Google Drive y habiendo aprendido de Google Calendar, se decidió no perder el tiempo buscando cómo crear las funciones desde cero, y se optó por descargar el ejemplo y coger lo necesario y adaptarlo a Baldugenda.
Cuando se importó el proyecto todo era muy confuso, y no se parecía a la forma de trabajar que se había seguido anteriormente en Google Calendar.
Aún y todo se miró el código, se entendió más o menos lo que hacía en cada función, y se intentó sacar las funciones necesarias al caso de uso de Backup de Baldugenda.
Cuando se sacaron las funciones y se comprobó que funcionaba la selección de ficheros por un lado, y la subida de ficheros por otro se intentó juntar.

El problema fue que Google al crear una carpeta mediante el \gls{activity}, que dan de ejemplo, pasa un tiempo hasta poder usar el identificador de esa carpeta. Y eso producía un fallo en el código, ya que al intentar subir a una carpeta que no tenía identificador fallaba el programa.
La solución fue separar por una parte la selección de la carpeta, y por otra la subida del fichero, con eso se le daba tiempo a Google a que generara el identificador, y de paso el usuario podía comprobar que la carpeta se había seleccionado correctamente.
Antes de optar por esa solución, se pensó usar las preferencias de la aplicación para guardar la dirección donde se guardaba la carpeta, y de esa manera poder realizar backups periódicos sin depender del usuario, pero el problema seguía siendo el mismo así que se descartó esa opción.

Cuando se implementó el backup no se tuvo en consideración el uso de distintos usuarios de Google en el mismo dispositivo. Así que al hacer las pruebas todo iba bien ya que no se modificaba la cuenta, en cambio al usuario sí que se le daba la posibilidad de cambiar las cuentas del Calendar para guardar la información. Al ser la forma de conexión distinta, no se consiguió usar los certificados creados para el Calendar dentro del Drive.
Se investigó por Internet y en StackOverflow, un usuario dio la solución que era agregando el API de Google plus, realizando un borrado del caché de los credenciales, y desconectando y volviendo a conectar, una vez realizado este procedimiento, volvía a pedir de nuevo el e-mail de la cuenta a la que se quería de nuevo acceder.

\newpage
\section{Android}
\label{secc:android}
Si se quiere desarrollar una aplicación en Android, éste es uno de los puntos más importantes de esta memoria, ya que se detallará aspectos principales y los problemas que se han tenido al desarrollar la aplicación y los que pueden surgir.
Para empezar, lo más importante ¿Por qué programar en Android y no en otro \acrshort{so}?
El motivo por el que Baldugenda está para Android  y no es multiplataforma, es que se quería 
comprobar el potencial que tiene Android y centrarse en esta plataforma.
Aparte si hablamos de números, podemos ver cómo están actualmente los sistemas operativos en el mercado móvil\cite{Tendencias}.

\begin{figure}[H] 
  \begin{center} 
    \scalebox{0.6}{\includegraphics{figs/TendenciasSO.png}} 
    \caption{Tendencias en móviles SO} 
    \label{fig:TendenciasSO} 
  \end{center} 
\end{figure}

Se puede ver que a principios de este año, Android le ha cogido la delantera a iOS, los dos son líderes en el mercado móvil.
Google ha conseguido resolver la fragmentación de sus versiones de Android y eso le ha dado dolores de cabeza a iOS, la cual ha pasado a ocupar el segundo lugar.
\newpage
\begin{figure}[H] 
  \begin{center} 
    \scalebox{1}{\includegraphics{figs/RankingSO.png}} 
    \caption{Ranking de móviles SO} 
    \label{fig:RankingSO} 
  \end{center} 
\end{figure}

El motivo anterior, sumado al aumento de personas que usan Android en sus dispositivos, y siendo yo un usuario de Android en estos momentos, me llevó a la decisión de querer sacarle el máximo partido realizando una aplicación nativa en Android.
Antes de comenzar a desarrollar en Android, hay ciertos aspectos que hay que saber.
Uno muy importante es que se necesita conocimientos de Java.
Estar acostumbrado a trabajar con programación orientada a objetos.
Aparte de estos dos anteriores, un IDE que actualmente el que más auge tiene entre los desarrolladores de Android es el Android Studio por su facilidad a la hora de crear un proyecto.
Si se tiene todo, lo anterior programar en Android no resultará complicado. Al principio parecerá todo un poco lioso pero el periodo de aprendizaje después compensa, ya que lo que al principio tardarías en hacerlo tres horas al cabo de dos semanas, podrás hacerlo en 10 minutos porque ya te sabrás la mayoría de funciones principales en Android.


\subsection{Restricciones de API}
\label{subsecc:Restricciones de API}

Ya se ha comentado anteriormente que Google ha ganado terreno en el mercado de los móviles por su labor en la desfragmentación de Android, aún con la labor de Google los desarrolladores nos seguimos encontrando que si no desarrollamos para las versiones antiguas de Android y solo desarrollamos para las nuevas, quitaremos una parte importante de los usuarios \cite{dashboards}.

\begin{figure}[H] 
  \begin{center} 
    \scalebox{1}{\includegraphics{figs/VersionesAndroid2.png}} 
    \caption{Versiones Android} 
    \label{fig:VersionesAndroid2} 
  \end{center} 
\end{figure}

Estas estadísticas son de mayo de 2015, si se quiere conservar a la mayor cantidad de gente, y no limitarse mucho con la aplicación, los expertos recomiendan empezar a olvidar a las personas que tiene versiones inferiores a Gingerbeard.
En la aplicación de Baldugenda, se cogió también a los de API 10, ya que las funciones que se iban a implementar no suponían gran avance tecnológico en cuanto a versión de Android, así que se podía usar perfectamente la versión 10 sin verse limitado.
Al escoger API inferiores, uno se expone a tener que realizar más código especial para esas versiones antiguas, e invertir tiempo en retocar aspectos para que no se note la diferencia y que el usuario no tenga problemas de interacción.

Cada una de las APIs mostradas arriba añade diferentes funciones, la más destacada, se puede ver comparando el API 10 con el API 21. El apartado gráfico que usa Android en una u otra varía mucho dependiendo de la versión que se use como mínimo.
Más adelante se explicará la interfaz en Android, pero para ver la diferencia y el coste que aumentaría usar una API de más bajo nivel del necesario, se puede ver en la barra de arriba de los dispositivos móviles. A partir del API 21 Android ha optado por un diseño llamado Material Design, este diseño produce que el usuario vea la aplicación más cercana al trabajar con sombras, distintas profundidades y colores más llamativos.
\newpage
El problema surge cuando se quiere usar este diseño en APIs más bajas, Google proporciona documentación para realizar el material design sobre versiones anteriores a Lollipop, pero aún y todo con toda esa información la diferencia entre usar un API como la 21 frente a usar un API 10 cambia mucho.

\begin{figure}[H] 
  \begin{center} 
    \scalebox{1.2}{\includegraphics{figs/AndroidToolbar.png}} 
    \caption{Android Toolbar colores android-developers} 
    \label{fig:AndroidToolbar} 
  \end{center} 
\end{figure}

En la imagen mostrada arriba se puede apreciar cómo funcionan los colores en Android desde la versión 21, en cambio si se quiere conseguir algo parecido a eso hay que implementar clases que hagan lo mismo, y no se puede basar sólo en las librerías de compatibilidad que ofrece Google.
Un componente que agregó el API 21 fue la toolbar que como se puede ver también permite cambiar el color de la barra donde se encuentra la hora, etc.
En el API 10 el resultado final es muy parecido, pero se invierte más tiempo que si el diseño no es primordial, molesta gastarlo agregando colores.

\subsection{Tipo de dispositivos}
\label{subsecc:Tipo de dispositivos}

Android a diferencia de IOS tiene una gama demasiado grande de dispositivos, muchas empresas usan el sistema operativo de Android para sus móviles.
Por este motivo cuando se quiere desarrollar para Android hay que saber muy bien a qué dispositivos se quiere dirigir la aplicación.
Los dispositivos Android pasan desde una pantalla de televisión, hasta un pequeño smartwatch.
Hay qué saber que dispositivos quitar para llegar a la mayor parte de los usuarios que se quiere que usen la aplicación.
Para resolver la duda Google ofrece periódicamente las estadísticas de los dispositivos usados y sus versiones.
De las versiones que ya se han hablado en un tema anterior, ahora en este apartado se hablará de las densidades y los tipos de pantalla.

\begin{figure}[H] 
  \begin{center} 
    \scalebox{0.7}{\includegraphics{figs/ComparativaPantallas.png}} 
    \caption{Comparativa tipo de pantallas} 
    \label{fig:ComparativaPantallas} 
  \end{center} 
\end{figure}

Estos datos los proporciona Google, actualizado en mayo de 2015.
Se puede apreciar las distintas gamas que hay de dispositivos. A la hora de crear el proyecto, si se usa Android Studio,el IDE dará consejos con estos temas preguntándote hacia qué dispositivo va destinada la aplicación y se tendrá  que elegir entre móviles o tablets, televisión, aparatos wear de Android, o las Google glass.
Dependiendo de qué elección se escoja en ese momento, se modificarán automáticamente los ajustes para que el proyecto esté preparado para el dispositivo seleccionado.
Baldugenda es una aplicación destinada a móviles y tablets, ya que su uso está pensado para ser algo habitual y de acceso rápido.
Las aplicaciones que se desarrollan del estilo de Baldugenda suelen ser para móviles también, ya que tienen que hacer la función de agenda virtual. Eso no quita para que también haya widgets de esas aplicaciones compatibles con dispositivos como los smartwatch donde se reciben las notificaciones de las aplicaciones de este género.

\subsection{Interfaz}
\label{subsecc:Interfaz}

Uno de los problemas más grandes que se encuentra uno al desarrollar en Android y en cualquier sistema operativo para dispositivos móviles es la limitación de espacio en la pantalla del aparato. También al haber distintos tipos de pantallas hacerlo compatible con todos es una ardua tarea.
Hay muchos aspectos que hacen que los usuarios no borren la aplicación, como puede ser que les parezca entretenida o útil, pero por mucho que una aplicación cumpla todo eso por debajo, a los usuarios se les gana por la vista. Si se quiere mantener la aplicación instalada en los dispositivos de los usuarios, ésta tiene que ser agradable para la vista y también cómoda de usar.
De algo que la mayoría de los informáticos pecamos, es que el diseño no es un aspecto importante dentro de nuestra labor, y por muy divertida que hagamos la parte lógica de la aplicación después el diseño se resiste a salir bien.

Pero aún y todo, evitando este tema, hay que dedicarle un tiempo después de haber finalizado la parte lógica y antes de publicar la aplicación al apartado de diseño.
Unos colores vivos frente a un fondo blanco marcan la diferencia.
En el caso de Baldugenda durante la primera fase se decidió hacer los casos de uso, y que fueran los Baldusers los que dieran propuestas de diseño que les parecieran más atractivas.
De esa forma se pasó de un menú principal de 6 botones grises y un fondo blanco a 6 iconos con fondo de colores.

Android proporciona muchos componentes para retocar la interfaz a la hora de desarrollar algunos tan vistosos como un ExpandableList, y otros que pasan desapercibidos como puede ser un TextView.
Lo más importante es sacarle provecho a la pequeña pantalla sobre la que se está trabajando, y usar dispositivos con pantalla pequeña para realizar las pruebas. Ya que estos casos son los más peligrosos, en el caso de que sobre mucho espacio en un dispositivo de tamaño más grande, siempre se podrá meter algo más dependiendo del tamaño, pero si el dispositivo tiene la pantalla demasiado pequeña, hay que saber comunicar al usuario para que no se pierda en la aplicación.

 
Algo muy vistoso y que no debe faltar en ninguna aplicación son los menús laterales, estos menús no ocupan espacio para el usuario, y cuando él quiere los abre y los cierra.
\newpage
\begin{figure}[H] 
  \begin{center} 
    \scalebox{0.7}{\includegraphics{figs/MenuLateral.png}} 
    \caption{Menu lateral Gmail} 
    \label{fig:MenuLateral} 
  \end{center} 
\end{figure}

En Baldugenda se puso un menú de este tipo en la lista de exámenes y les pareció interesante a los usuarios.
Algo que como desarrolladores hay que aprender, es que si se quiere publicar la aplicación, y que pueda ser descargada, tiene que tener funcionalidades que a los usuarios les parezcan interesantes, y por ello deben ser los propios usuarios los que den ideas de nuevas funcionalidades para futuras mejoras. La aplicación tiene que ir destinada a los usuarios, y por este motivo ellos no se van a fijar en la dificultad que tiene sacar la información de la base de datos, y mostrarle justo lo que buscan cuando lo buscan.
A ellos lo que les importa principalmente es que sea impactante y divertida a la hora de interactuar con ella.
Si se cambia un botón por algo distinto donde el usuario arrastre o realice otra acción eso le sorprenderá.

El ejemplo que se encontró en Baldugenda es al realizar borrados y modificaciones.
Dependiendo del usuario que la ejecutaba y la edad, que tuviera, estaban más acostumbrados a las acciones especiales de los botones como mantener pulsado o arrastrar.
Un punto interesante también son los Dialogs, ventanas que se muestran superpuestas a la actividad que se está ejecutando. Gracias a esos dialogs se gana mucho espacio y se consigue que el usuario preste atención a la zona donde se quiere en cada momento.

Volviendo a las pulsaciones largas y juntándolo con el tema de los dialogs en Baldugenda, se juntó todo esto y usaron los menús para realizar acciones sobre un objeto en específico.
Los expandable list view es una buena forma de mostrar listas muy largas de lo que se precise agrupándolas por un motivo en concreto.
De esta forma el usuario no tendrá una lista que le ocupa la pantalla, en vez de eso tendrá cajitas, que abriendo una tendrá lo que busca de esa categoría.


\begin{figure}[H] 
  \begin{center} 
    \scalebox{0.7}{\includegraphics{figs/ExpandableList.png}} 
    \caption{Expandable list view} 
    \label{fig:ExpandableList} 
  \end{center} 
\end{figure}

En este punto también entra el toque artístico de cada persona, cada caja se puede diseñar de la forma que el desarrollador quiera y también añadir los colores.

\subsection{Migración}
\label{subsecc:Migración}

Cuando se decidió implementar en Android, se descartó usar aplicaciones para desarrollar multiplataforma.
Por este motivo hubo que mirar las posibilidades que había, y las dificultades que se encontraría si se tuviera que implementar Baldugenda, por ejemplo para IOS o Windows Phone.
Microsoft se ha adelantado a la compañía del Androide y de la manzana, y ha optado por ayudar a los desarrolladores a migrar aplicaciones desde IOS y Android a Windows Phone.
Esta medida se debe a las estadísticas que se han podido ver sobre los smartphones y el tipo de sistema operativo que usan.

Para ello ha ideado una API estilo diccionario para ver las equivalencias entre Android y Windows phone y con eso hacer más fácil a los desarrolladores la conversión de las clases y las funciones.
Entre IOS y Android no es tan fácil la migración. Los dos son empresas muy fuertes en el mercado de los móviles y no darían su brazo a torcer para ayudar a su competidora.
Pero algo positivo que tiene trabajar con un modelo MVC es que cada apartado está separado.
El modelo se podría trasladar directamente a IOS y la parte de la vista y controlador se podría distribuir entre más personas para desarrollar sobre el ejemplo de Android.
Cuando se habla de migraciones siempre viene a la cabeza el cambio de sistema operativo y aspectos así, pero también una migración podría ser el salto de una aplicación móvil a un dispositivo wear de Android o a una pantalla de televisión. Para eso, el cambio vendría a ser muy parecido pero pudiendo usar gran parte del código exceptuando la parte de la vista y la conexión entre la vista y la lógica.

Un punto importante que hay que tener en cuenta a la hora de migrar no es solo instalar la aplicación en un nuevo dispositivo, hay que pensar en los usuarios y permitirles llevarse todo lo realizado hasta el momento de la aplicación anterior a la nueva.
Hay aplicaciones que guardan la información en sus propios servidores, y ese paso es transparente para el usuario.
En cambio en el caso de Baldugenda al no usar ningún servidor exceptuando el de Google Calendar para guardar los exámenes, se decidió poder migrar los datos mediante la opción de exportar base de datos y usar la cuenta de Google para pasar de un dispositivo a otro.

\subsection{Notificaciones}
\label{subsecc:Notificaciones}

Las notificaciones en Android es un componente importante dentro de las aplicaciones. Es la forma con la que la aplicación se comunica con el usuario.
Hay distintos tipos de notificaciones:
Están las notificaciones Toast que son mensajes que se escribirán en la pantalla durante un periodo largo o corto según se le indique.
\newpage
\begin{figure}[H] 
  \begin{center} 
    \scalebox{0.7}{\includegraphics{figs/NotificacionToast.png}} 
    \caption{Notificación Toast} 
    \label{fig:NotificacionToast} 
  \end{center} 
\end{figure}

A la hora de realizar la aplicación si no se encuentra dónde está el fallo, una forma muy útil de usar las notificaciones Toast es sacar por este mensaje los valores que nos interesen para saber si se está trabajando sobre el objeto que se necesita.
Dentro de Baldugenda se ha usado los Toast para comunicar al usuario que se había creado un examen o una asignatura. Hay veces que el teclado tapa estos mensajes, así que hay que tener cuidado en qué momento se escriben porque puede que el usuario no lo vea.

Otro tipo de notificaciones son los alertdialog y los timepicker dialog y datepicker dialog.
El alertdialog admite un título, un texto y como máximo tres botones. Se suele usar para hacer que el usuario confirme una acción, en Baldugenda se ha usado a la hora de borrar la asignatura, se le pregunta si está seguro de borrarla, si pulsa que sí la asignatura se borrará en cambio si cancela no habrá pasado nada.
\newpage
\begin{figure}[H] 
  \begin{center} 
    \scalebox{0.7}{\includegraphics{figs/EjemplosDialog.png}} 
    \caption{Ejemplos de dialog} 
    \label{fig:EjemplosDialog} 
  \end{center} 
\end{figure}

Tanto el timepicker dialog como el datepicker dialog son ventanas modales que actúan sobre la hora y la fecha respectivamente.
Se le da la opción al usuario que escoja el día mediante el datepicker y mediante el timepicker puede seleccionar la hora.

Dependiendo de la versión del dispositivo estas ventanas modales serán de una forma u otra.

\begin{figure}[H]
 \centering
  \subfloat[Ejemplo DatePicker en Lollipop]{
   \label{f:Ejemplo DatePicker en Lollipop}
    \includegraphics[width=0.4\textwidth]{figs/DatePickerLollipop.png}}
  \subfloat[Ejemplo DatePicker en JellyBean]{
   \label{f:Ejemplo DatePicker en JellyBean}
    \includegraphics[width=0.4\textwidth]{figs/DatePickerJellyBean.png}}
 \caption{Tipos de DatePicker dependiendo de la version}
 \label{f:Tipos de DatePicker dependiendo de la version}
\end{figure}

A la izquierda podemos ver el datepicker en la versión de lollipop y a la derecha un una versión anterior.
Para el manejo de los días se usa la clase Calendar de java, y eso ha venido muy bien ya que Google Calendar, sus funciones permiten directamente meter una fecha creada en un objeto Calendar al evento.
De esa forma, el trabajo que hubiera supuesto tener que sacar la fecha actual, calcular cuánto falta hasta la fecha elegida por el usuario y demás, se quita, y sólo se tuvo que sacar la información que seleccionaba el usuario e insertarla en el evento.

Aparte de estas notificaciones, tenemos una muy importante si se va trabajar con servicios de Google, que es el progress dialog.
Este tipo de componente es muy útil con estos servicios ya que al usar los servicios de Google hay veces que las conexiones tienen que hacerse en segundo plano, y hay que lanzar tareas asíncronas, pero se sigue dependiendo del resultado del servicio para seguir trabajando.
Con este componente se genera una barra de progreso que se irá llenando según avance la tarea asíncrona, o directamente un mensaje que no dejará realizar ninguna otra acción hasta la finalización de la tarea asíncrona.
En Baldugenda se le ha dado uso al realizar la búsqueda de los calendarios y al realizar la creación de exámenes, porque dependiendo de la velocidad que tenga el móvil por Internet, puede que el servicio de Google Calendar no vaya lo más rápido que debería, y dé error si dejamos que se genere en la ventana secundaria sin avisar al usuario. 

\begin{figure}[H] 
  \begin{center} 
    \scalebox{0.8}{\includegraphics{figs/ProgressDialog.png}} 
    \caption{Ejemplos de ProgressDialog} 
    \label{fig:ProgressDialog} 
  \end{center} 
\end{figure}

Mostrarle al usuario una ventana explicándole lo que está pasando en cada momento hace que no se ponga a pulsar todos los botones pensando que se le ha quedado parada la aplicación.
Se han dado situaciones en el proyecto que por no avisar al usuario que se había generado un examen mediante una ventana de progress dialog, el usuario se pensaba que no estaba generada, y la intentaba crear otra vez con el consiguiente error.
Por eso es una buena opción si se quiere llamar la atención del usuario, si lo comparamos con los toast, esta opción es más engorrosa pero el usuario la verá aunque tenga el teclado abierto.

\subsection{Uso de calendarios}
\label{subsecc:Uso de calendarios}

Ya se ha hablado de los timepicker dialog y datepicker dialog en el apartado de las notificaciones de Android.
En este punto se hablará más a fondo del uso que se le suelen dar a los calendarios en las aplicaciones del tipo Baldugenda.
Android tiene un calendario dentro del dispositivo que si no se le asocia ninguna cuenta trabajará con eventos en local. Muchas aplicaciones usan este calendario para generar ahí los eventos, otras simplemente crean su propio calendario dentro de la aplicación.
El problema que se vio al generar el calendario dentro de la aplicación, era que el usuario tenía que estar duplicando los eventos si quería tenerlo siempre disponibles.
Una de las mejoras que se tiene pendiente en Baldugenda es la visualización de los eventos en formato mes.

\begin{figure}[H] 
  \begin{center} 
    \scalebox{0.8}{\includegraphics{figs/Caldroid.png}} 
    \caption{Ejemplo de Caldroid Github} 
    \label{fig:Caldroid} 
  \end{center} 
\end{figure}

Como se puede observar en la imagen, esa sería un tipo de ventana que se quiere implementar en la aplicación Baldugenda para mostrar por días los exámenes que se tienen apuntados.

Por falta de tiempo no se realizó, aunque sí se tenían propuestas librerías para realizar este tipo de vista como puede ser Caldroid, una librería con copyright disponible en Github que ofrece las funciones necesarias y el diseño para mostrar el calendario de esta forma, y poder generar eventos y marcarlos de colores.
Aún y todo se seguiría usando Google Calendar para mantener consistencia a los calendarios del usuario, y que tenga acceso siempre a los eventos que se generan dentro del móvil.
Android ofrece su propio componente de vista de calendario, el Calendar view, el problema que se encontró al usarlo es las limitaciones visuales que tiene, se buscaba al usar esa vista, mostrar los exámenes que había en el mes, y con Calendar view no se podía agregar eventos a los días del calendario.

Por ese motivo, se decidió dejar el tema de la vista del calendario como algo secundario y centrarse en funcionalidades más importantes como el backup.
Se ha probado a instalar muchas aplicaciones que tienen la finalidad de guardar asignaturas y exámenes en el móvil, junto con sus fechas, pero el resultado siempre era el mismo exceptuando alguna que era de pago, y que en su versión gratuita no permitía hacerlo, pero en la versión de pago sí. Las demás aplicaciones funcionaban en local y algunas daban la posibilidad de exportar los calendarios a la tarjeta sd del móvil por si quería usarlos.
De esta forma se propuso la idea de vincular Google Calendar a la aplicación y darle un toque novedoso.

\subsection{Tareas asíncronas}
\label{subsecc:Tareas asíncronas}

Este punto es muy importante ya que a partir de la versión cuatro no se permite realizar peticiones http en el hilo principal de la aplicación, esto tiene sentido ya que significaría ralentizar la aplicación llegando en algunos momentos a bloquearla.
Para esto se crean clases asíncronas que se lanzaran con las llamadas a los servicios que se necesiten.
\newpage
\begin{figure}[H] 
  \begin{center} 
    \scalebox{0.8}{\includegraphics{figs/TareasAsincronas.png}} 
    \caption{Tareas asíncronas Android funcionamiento arquitecturajava.com} 
    \label{fig:TareasAsincronas} 
  \end{center} 
\end{figure}

En el apartado de notificaciones se le ha dado importancia a la parte del progres dialog, esto es porque mientras se está ejecutando la tarea asíncrona, la actividad sigue funcionando en un thread principal. Por este motivo, si se necesita esperar a que la tarea finalice y devuelva el dato a la actividad, habría que usar ese progres dialog para que no se permita modificar nada hasta que acabe.
Al extender de la clase AsynTask de Android, obligará a que se tenga que implementar la función de doInbackground, esta función es la que realizará la tarea principalmente.

Aparte de esta función, hay distintas funciones que serán de utilidad para trabajar con la tarea.
Una de las funciones es onPreExecute, si se implementa, es la función que se realizará antes de entrar al nuevo thread, desde esta función todavía se puede modificar la parte visual de la actividad donde se ha lanzado la asynctask. Una vez se pasa al doinbackground ya no se podrá modificar.
En la función anterior es donde se suele generar el progresdialog.

Después, si se quiere modificar los valores de la UI de la actividad estando desde el asynctask, se necesita recurrir a dos funciones, publishProgress y onProgressUpdate.
Esas dos funciones serán las encargadas de ir modificando el valor, y pasando los datos que quieren que se muestre en cada momento en la ventana de progreso que se ha lanzado en la función onPreExecute, también pueden usarse para modificar información que tengan que ver con la UI principal.
Y para terminar, una de las funciones que más se usa es la de onPostExecute, esta función es la que se lanzará cuando se ha terminado  la función doInBackground, si se ha añadido el progres dialog hay que quitar el objeto en este punto ya que sino, no se podrá tocar de nuevo la pantalla de la actividad principal.
A continuación se muestra el ciclo de vida de una asynctask

\begin{figure}[H] 
  \begin{center} 
    \scalebox{0.8}{\includegraphics{figs/CicloVidaAsynctask.png}} 
    \caption{Ciclo de vida de Asynctask} 
    \label{fig:CicloVidaAsynctask} 
  \end{center} 
\end{figure}

\subsection{Backup}
\label{subsecc:Backup}

Aunque ya se ha hablado de este tema en el apartado de Google, el tema del backup en Android es un asunto muy importante, la mayoría de aplicaciones tienen algún método para guardar la información que genera el usuario.
Algunas aplicaciones hacen uso del API comentada anteriormente para realizar el backup que ofrece Google, la ventaja de esta propuesta es que la información estará guardada y se podrá recuperar la siguiente vez que el usuario quiera instalar la aplicación.

Otras aplicaciones realizan los backups directamente sobre el dispositivo, usando la tarjeta SD para guardar la información, y de esta forma en el caso que el dispositivo se averíe hay una base de datos consistente que se podrá importar.
Al realizar la aplicación, en los primeros casos de uso no se le dio importancia a los backups, pero después de circunstancias como fue en mi caso que se me estropeó el móvil, se echa en falta. La buena noticia fue que al tener vinculado con Google Calendar todos los exámenes se podía saber la fecha que tenían al mirar por la web.

Para que una aplicación sea útil, tiene que permitir al usuario llevarse la información entre dispositivos, y si por alguna circunstancia se desinstala la aplicación, que pueda instalarla de nuevo con la información que tenía anteriormente.
Hay aplicaciones que usan cuentas de usuario para volver a rellenar la base de datos en local del móvil, pidiéndole al servidor al momento de instalarlo, y así no tener que estar haciendo consultas en cada momento.
Lo que se vio como una buena oportunidad era juntar el uso de un API de Google junto con el backup de la aplicación. Por eso se decidió usar Google Drive.

\subsection{Debugging}
\label{subsecc:Debugging}

Cuando se está desarrollando cualquier proyecto, la mejor manera de ver si funcionan las implementaciones es probándolas, en el caso de Android es una de las mejores opciones para que todo funcione como se espera.
Los dispositivos móviles tienen muchas variables que pueden afectar a la aplicación, las versiones o la pantalla son las más generales. Pero no son las únicas, el acceso a Internet para una aplicación que tiene que estar sincronizada a los calendarios del usuario es un factor determinante para que vaya bien.

La posición en la que se use el dispositivo suele ser un tema complicado de llevar, por eso el apartado de debugging es de los temas más importantes que hay que realizar a la hora de querer publicar una aplicación en el Google Play.
Las aplicaciones que se cuelgan al Google Play no suelen pasar por muchos controles de calidad, así que el desarrollador es el que tiene que ser el control de calidad de su aplicación. Si el propio desarrollador no está satisfecho con su trabajo, es que todavía la aplicación no está disponible para salir al mercado.
Si se combina la depuración del programa, junto con usuarios que puedan probarla y devolver feedback, la aplicación saldrá mucho mejor de lo esperado, y mucho antes que si el propio desarrollador tiene que estar haciendo todas las pruebas.

Hay aspectos que hay que probar cuando se hace el debugging si se está trabajando con un móvil. Lo mejor es ir realizando código pequeño y probarlo poco a poco en la aplicación, en vez de lanzarse a programar toda la aplicación de golpe y probarlo después, en Android los errores pueden salir por muchos sitios, desde una línea no escrita en el manifest o un botón no declarado, a un fallo de asignación en el TextView que se piensa que se está insertando un String y al final era un integer.

Por todos estos fallos pequeños de programación si se realiza una aplicación con varios casos de uso, el error puede surgir en cualquier función, por ese motivo es más sencillo realizarlo de la forma siguiente: lo primero, realizar la distribución de la vista que tendrá el usuario y luego probar que se vea adecuadamente.

Una vez que se puede ver correctamente en todas las posiciones, ya se puede pasar a darles utilidad por ejemplo a los botones. Probar con una notificación estilo Toast que funcionan correctamente los botones, y que si se ha programado que salte a otra actividad lo hace sin problemas. También comprobar que pasa al darle hacia atrás, y si es lógico que haga eso, ya que las actividades en Android tienen su ciclo de vida e igual no interesa que sigan vivas una vez se ha realizado alguna acción.
Cuando ya las acciones simples funcionan, lo siguiente es probar si la función que se ha implementado funciona correctamente, para ello en el caso que esa función devuelva un valor, lo más rápido es añadir al Toast anterior el valor de la función, y añadir un breakpoint en la función a la hora de ejecutar las pruebas conectado al ordenador.
Con ese breakpoint cuando se lance la aplicación en la máquina virtual, o el dispositivo físico, podrás ejecutar el botón y se parará cuando entre en la función marcada. Se puede ir línea por línea e ir viendo qué valores están cogiendo las variables.
Lo mejor en mi opinión para hacer pruebas y no estropear la batería ni el enchufe del móvil es usar una máquina virtual de Android. El problema es el consumo de memoria que tiene el uso de esta opción. Android Studio ofrece su propio administrador de máquinas virtuales de Android. Pero recomiendo usar un software que se llama Genymotion que también está adaptado a Android Studio, la ventaja de este software es que no consume tantos recursos y el móvil va más fluído.

En el caso de que se trabaje con servicios de Google las máquinas virtuales no traen por defecto los servicios de Google instalado, así que hay que instalarlo en la máquina como si fuera software de terceros. Hay tutoriales por Internet e incluso webs que se dedican a subir las apks de Google play para que se puedan instalar dependiendo la versión del móvil.
El funcionamiento es sencillo, solo hay que instalar 2 ficheros uno es el ARM Translation installer, y el otro son los servicios de Google donde se encuentran Gmail,Google maps, youtube, etc.
Cuando se realizan estos pasos ya se tendrá un móvil Android virtual como el que se tiene físico.
Al trabajar con una máquina virtual, las capturas de pantalla y grabaciones son sencillas de hacer. El acceso a las carpetas donde se guardan las bases de datos de la aplicación dispone de permisos de super usuario para entrar y descargar esos datos, cosa que si se prueba en un dispositivo físico se necesitaría rootear el teléfono para hacer lo mismo.

Una vez que los casos de uso funcionen como se tenía pensado, es hora de subir la aplicación en versión alpha y hacer que los usuarios la prueben.
Para ello algo, útil es hacerles un guion a la hora de realizar tareas, ver si lo pueden realizar y si les da algún tipo de fallo.
Aparte del guión es útil que ellos vayan usándola como lo usarían de normal para ver los posibles errores que puedan salir si se lanza al mercado en este momento.
Mediante software que recoge la información de la aplicación como puede ser Google Analytics o Splunk Mint se va revisando los fallos que puede estar dando la aplicación.

Durante las pruebas de Baldugenda se usó Splunk Mint, un software que todavía está en desarrollo. Pero que funciona muy bien para lo que se necesitaba de la aplicación, que era captar los errores y decir dónde fallaba sin necesitar que el usuario dijera cómo le había fallado.
Para la instalación dentro del terminal es muy sencilla, lo único que hay que hacer es registrarse en Splunk Mint y una vez registrado crear un proyecto, agregar la librería mediante gradle o metiéndola en el proyecto.
Y después seguir los pasos que marcan en la web para hacer las llamadas a sus servicios.
Ofrece un seguimiento muy bueno dentro de la aplicación, se puede programar para que cuando un usuario pulse un botón esa acción quede recogida en su página web, y después lo mostrará cuando entremos. También realiza notificaciones al e-mail que le digamos.

El manejo es muy sencillo, se separa en tres pestañas, la primera es donde está la información de las conexiones y de los dispositivos que tienen instalada la aplicación, la segunda pestaña es para los errores que se dan y te permite filtrarlos por fecha, y te muestra un gráfico de tiempo la cantidad de errores que se han producido y la versión que en la que se ha producido el error. Y la tercera pestaña es para las configuraciones del proyecto, Splunk mint permite vincular con el proyecto que se suba a Github y hacer comentarios dentro del código, en el caso que se resuelva algún error se le puede decir a Splunk Mint que escriba en Github que se ha resuelto y en qué versión ya no se produce.

\begin{figure}[H] 
  \begin{center} 
    \scalebox{0.8}{\includegraphics{figs/SplunkMintMenu.png}} 
    \caption{Splunk Mint Menú} 
    \label{fig:SplunkMintMenu} 
  \end{center} 
\end{figure}

Los errores aparecen de la siguiente manera, y es posible entrar a ellos y ver línea a línea por dónde ha pasado el programa hasta llegar al error.

\begin{figure}[H] 
  \begin{center} 
    \scalebox{0.8}{\includegraphics{figs/ErrorSplunkMint.png}} 
    \caption{Error en Splunk Mint} 
    \label{fig:ErrorSplunkMint} 
  \end{center} 
\end{figure}

También se puede configurar para que envié los errores al e-mail que queramos.

\begin{figure}[H] 
  \begin{center} 
    \scalebox{0.8}{\includegraphics{figs/ConfiguracionSplunkMint.png}} 
    \caption{Configuración  Splunk Mint} 
    \label{fig:ConfiguracionSplunkMint} 
  \end{center} 
\end{figure}

Se trabaja muy bien en el caso que sea un grupo de personas, ofrece la posibilidad de permitir ver los errores a un equipo de proyecto y enviar las notificaciones.

\subsection{Control de versiones}
\label{subsecc:Control de versiones}

Al desarrollar en Android una parte importante es mantener siempre accesible el código y saber qué cambios se han realizado, un programa, da igual el lenguaje que se esté usando y el dispositivo al que vaya enfocado es igual que un libro, no se redacta o en el caso de aplicaciones se desarrolla de un día para otro. Por este motivo es útil siempre usar el control de versiones para tener constancia de los cambios efectuados y si en algún momento se necesita volver a una versión anterior, es más fácil realizarlo que empezar a escribir todo de nuevo, pudiendo ocasionar perdida de código por el camino y el fallo del programa.

En el caso de Android, su IDE, Android Studio facilita enormemente el trabajo con el control de versiones, ya que trae incorporado opciones que permiten subir directamente el código a Github. El único requisito que pide es tener una cuenta de Github y ya estaría funcionando la subida de código con su respectivo control de versiones.
Los puntos positivos de usar el control de versiones de Github es que aparte de saber el propio desarrollador lo que ha ido generando día tras día, es un buen método para que las demás personas conozcan tus trabajos y vean, si eres una persona periódica en las tareas de programación.

El propio Google usa Github para tener sus proyectos de ejemplo de los APIs que distribuye. Así que habituarse a usar esta herramienta es un buen comienzo para ver cómo trabajan los demás desarrolladores.
Aparte Github tiene plugin para múltiples programas como puede ser los de procesador de latex, una manera útil de realizar la memoria y subirla al repositorio y tenerla segura.
En el caso de usar Word 2013, ofrece también la posibilidad de marcar el control de cambios realizados, no es tan útil como Github, que aparte de guardarte los cambios, puedes viajar por las versiones del fichero subido, pero por lo menos se tendrá constancia de los cambios realizados ese día.

Igual que Github hay otros repositorios interesantes como puede ser gitlab, gitorious o bitbucket.
Cada uno tiene sus ventajas con respecto a los otros. Github por ejemplo tiene solo repositorios públicos si se quiere usar de forma gratuita, en el caso que se quiera repositorios privados hay que pagar.
En bitbucket y los rivales de Github aprovechan eso de los repositorios privados, y ofrecen uno o dos repositorios privados de forma gratuita al crear la cuenta.
De esta forma los desarrolladores pueden estar desarrollando código sin la sensación que la competencia se lo pueda quitar.
Cuando se está implementando código, una buena forma de hacerlo es ir subiendo el código que funcione únicamente, por mucho que se estén haciendo pruebas, el repositorio no tiene que ser un lugar donde se guarda el código que igual un día se use.
La forma de ver el repositorio tiene que ser un lugar donde guardar el código que ya se ha probado y que funciona, y poco a poco ir aumentándolo hasta tener un programa consistente que aunque dé fallos en ciertos momentos, por lo menos son fallos de las últimas modificaciones realizadas.
También se puede ver cómo una manera de acceder al código sin tener que llevarlo siempre en el USB, puedes estar trabajando en el ordenador de la facultad y cuando se acaba subirlo al repositorio, al llegar a casa descargarlo y seguir trabajando.

En la situación de Baldugenda no era preciso usar Github como método de trabajo en equipo ya que el equipo estaba formado por una sola persona, pero aún y todo, era una buena forma de llevar al día el tiempo que había llevado hacer cada implementación y que se había ido haciendo en cada fase del proyecto.
Si una aplicación la van a realizar más personas, es casi imprescindible el uso de herramientas como Github, ya que no se puede estar exportando el proyecto y compartiéndolo por correo con cada modificación que se haga, porque se produciría una confusión enorme.

Un problema de usar Github es que todos los miembros del equipo tienen que saber usarlo, sino puede llevar a problemas de código corrupto al subirse, o fallos que no permitirían subir el código. Pero una vez aprendido a usar Github, el uso que se le puede dar es enorme, ahora permite crear páginas web explicando el proyecto que está subido y permite que otros se lo descarguen o que hagan copias del proyecto en sus repositorios.
En el caso que no se utilice el plugin de Eclipse o Android Studio el funcionamiento de Github es por la consola. Se ha desarrollado aplicaciones propias para Mac  y para Windows, que hacen más fácil e interactivo la subida y descarga de los proyectos.

\subsection{Problemas al desarrollar en Android}
\label{subsecc:Problemas al desarrollar en Android}

Ya se ha hablado de los puntos importantes al crear un proyecto en Android y aspectos que hay que tener en cuenta.
Al desarrollar en Android hay ciertos momentos que aparecen fallos y no se encuentra solución, en este apartado se hablará de los casos que han sucedido en Baldugenda y cómo se han solucionado.
Uno de los casos más comunes es cuando se importa un proyecto de Github o eclipse. La mayoría de veces el IDE no detecta bien las librerías asociadas al proyecto, y marca errores tanto de vistas como de la clase R de Android.
Algunas maneras de solucionarlo es instalando las APIs que requiere el proyecto, al no tenerlas descargadas e instaladas en el ordenador donde se está implementando, esto ha podido hacer que falle la compilación, y ha hecho que falle Android Studio.
Otra solución desesperada es dentro del menú build de Android Studio la opción clean, con esto se limpiará la compilación anterior y realizará una nueva.

Un problema muy común para los que desarrollan en Eclipse, son las modificaciones en el manifest o xml de vistas, hay muchas veces que se genera una actividad nueva y se olvida que para usarla tiene que estar en el manifest apuntada.
Con Android Studio este problema no pasa en gran medida, ya que da la opción de generar todo de golpe, genera los layouts, modifica el manifest y genera los menús si se quiere sólo dándole a un botón.
Si se es nuevo con Android, como era yo al empezar la aplicación, no se conoce tan a fondo los entresijos  que depara Android a la hora de programar, uno de ellos es el uso de los servicios web.

Acostumbrado a realizar las llamadas de los servicios web donde quisiera, un error fue meterlo directamente en el hilo principal de la aplicación, al ejecutar la aplicación se bloqueaba, y el debugging llevaba a unas clases creadas por Android.
Buscando por Internet explicaban que en Android toda acción que haga que la pantalla se bloquee durante un tiempo no estaba permitida hacerla en el hilo principal de la actividad.
Ese descubrimiento fue muy útil, ya que con eso aprendí  mucho sobre los hilos en Android y las tareas asíncronas.

Antes se ha hablado del uso de máquinas virtuales, uno de los problemas que tiene usar máquinas virtuales son los recursos del ordenador físico que las está moviendo. En más de una ocasión se ha quedado el ordenador sin memoria por tener el Android Studio y la máquina virtual abierta, el único consejo que hay al respecto es intentar no usar máquinas virtuales muy pesadas, cuanto más alta es la API y mayores recursos se le asignan a la máquina más pesada es, y más va a exigir.
Android Studio es un IDE potente para desarrollar una aplicación Android, el problema de esto es que al ser potente también exige mucho. Si se quiere desarrollar en Android se necesita un equipo que tenga suficiente memoria para no quedarse esperando porque se ha quedado parado al compilar el Gradle.




% Mediante los siguientes comandos, podrás compilar el conjunto de ficheros desde este mismo documento


%%% Local Variables: 
%%% mode: latex
%%% TeX-master: "../Principal"
%%% End: 


















% line in order to check if utf-8 is properly configured: áéíóúñ